\documentclass[10pt, landscape]{report} % landscape for wider columns
\usepackage[a4paper,margin=0.2in]{geometry} % tight margins for dense layout
\usepackage{multicol}                     % multiple columns
\usepackage{amsmath,amssymb,mathtools,bm} % math packages
\usepackage{physics}                      % handy for derivatives, norms, etc.
\usepackage{booktabs,tabularx}            % tables
\usepackage{tikz,pgfplots}                % diagrams or plots (optional)
\usepackage{microtype}                    % better text spacing
\usepackage{enumitem}                     % compact item lists
\usepackage{fancyhdr}                     % custom header/footer
\usepackage{hyperref}                     % clickable refs
\usepackage{titlesec}                     % section spacing control
\usepackage{siunitx}                      % nice number formatting
\usepackage{xcolor}                       % color for highlights
\usepackage{tcolorbox}                    % boxed formulas
\usepackage{parskip}                      % spacing between paragraphs
\usepackage{amsthm}

% -------------------------------------------------------------
% PAGE & TEXT DENSITY SETTINGS
% -------------------------------------------------------------
\setlength{\parindent}{0pt}
\setlength{\parskip}{0pt}
\setlist[itemize]{noitemsep, topsep=1pt, left=1mm}
\setlist[enumerate]{noitemsep, topsep=1pt, left=1mm}
\setlength{\columnsep}{0.2in} % space between columns
\renewcommand{\baselinestretch}{0.92} % slightly denser lines
\renewcommand{\familydefault}{\sfdefault} % clean sans-serif font for readability

% Adjust section titles for compactness
\titleformat{\section}{\bfseries\large\color{blue!70!black}}{}{0em}{}
\titleformat{\subsection}{\bfseries\normalsize\color{teal!70!black}}{}{0em}{}
\titlespacing*{\section}{0pt}{1ex}{0.5ex}
\titlespacing*{\subsection}{0pt}{0.5ex}{0.25ex}

% -------------------------------------------------------------
% HEADER & FOOTER
% -------------------------------------------------------------
\pagestyle{fancy}
\fancyhf{}
\lhead{\textbf{Machine Learning Cheat Sheet — Open Book}}
\rhead{\today}
\cfoot{\thepage}

% -------------------------------------------------------------
% MATH SHORTCUTS
% -------------------------------------------------------------
\newcommand{\R}{\mathbb{R}}
\newcommand{\E}{\mathbb{E}}
\newcommand{\Var}{\mathrm{Var}}
\newcommand{\Cov}{\mathrm{Cov}}
\newcommand{\KL}{\mathrm{D_{KL}}}
\newcommand{\N}{\mathcal{N}}
\newcommand{\softmax}{\mathrm{softmax}}
\newcommand{\sigmoid}{\sigma}
\newcommand{\argmin}{\operatorname*{arg\,min}}
\newcommand{\argmax}{\operatorname*{arg\,max}}
\newcommand{\ind}{\mathbf{1}}

% -------------------------------------------------------------
% BOXED FORMULAS (optional)
% -------------------------------------------------------------
\tcbset{
  colback=white,
  colframe=blue!50!black,
  boxrule=0.4pt,
  arc=2pt,
  left=3pt,
  right=3pt,
  top=2pt,
  bottom=2pt,
  fonttitle=\bfseries\color{blue!60!black}, % <- use fonttitle directly
  coltitle=blue!60!black                    % <- sets title text color safely
}


\newtcolorbox{formula}[1][]{
  colback=blue!2!white,
  colframe=blue!50!black,
  title=#1,
  boxrule=0.3pt,
  sharp corners,
  fontupper=\footnotesize
}

% -------------------------------------------------------------
% CUSTOM SMALL FONTS FOR CHEAT SHEET
% -------------------------------------------------------------
\newcommand{\tinyfont}{\fontsize{7.5pt}{8.5pt}\selectfont}
\newcommand{\smallfont}{\fontsize{8.5pt}{9.5pt}\selectfont}

\theoremstyle{plain}
\newtheorem{theorem}{Theorem}[section]         % Theorems numbered within sections
\newtheorem{lemma}[theorem]{Lemma}             % Lemmas share numbering with theorems
\newtheorem{proposition}[theorem]{Proposition} % Propositions share numbering
\newtheorem{corollary}[theorem]{Corollary}     % Corollaries share numbering

\theoremstyle{definition}
\newtheorem{definition}[theorem]{Definition}   % Definitions have upright text
\newtheorem{example}[theorem]{Example}         % Examples same numbering style

\theoremstyle{remark}
\newtheorem{remark}[theorem]{Remark}           % Remarks in italic header but normal text
\newtheorem{note}[theorem]{Note}



\begin{document}
% ============================================================
\chapter{Linear Regression}
\begin{multicols}{2}
\section*{Ordinary Least Squares}
\subsection{}
\begin{theorem}[Uniqueness of the Least Squares Solution]
Let $\Phi \in \mathbb{R}^{N \times M}$ denote the design matrix and $t \in \mathbb{R}^N$ the target vector.
Consider the least squares cost function
\[
E(w) = \frac{1}{2}\|t - \Phi w\|^2.
\]
Then:
\begin{enumerate}[label=(\roman*)]
    \item The function $E(w)$ is convex in $w$.
    \item If $\Phi^\top \Phi$ is invertible (i.e., $\mathrm{rank}(\Phi) = M$), then $E(w)$ is \emph{strictly convex} and admits a unique minimizer
    \[
    w^* = (\Phi^\top \Phi)^{-1}\Phi^\top t.
    \]
    \item If $\Phi^\top \Phi$ is singular, the minimizer is not unique; all minimizers are of the form
    \[
    w = w_0 + v, \qquad v \in \mathrm{Null}(\Phi),
    \]
    where $w_0$ is any particular solution to the normal equations
    $\Phi^\top \Phi w = \Phi^\top t$.
\end{enumerate}
\end{theorem}

\begin{proof}
We begin by expanding the objective:
\[
E(w) = \frac{1}{2}(t - \Phi w)^\top (t - \Phi w)
      = \frac{1}{2}(t^\top t - 2t^\top \Phi w + w^\top \Phi^\top \Phi w).
\]

\noindent\textbf{(1) Gradient and Stationary Point:}
The gradient of $E(w)$ with respect to $w$ is
\[
\nabla_w E(w) = -\Phi^\top t + \Phi^\top \Phi w.
\]
Setting $\nabla_w E(w) = 0$ yields the \emph{normal equations}
\[
\Phi^\top \Phi w = \Phi^\top t. \tag{1}
\]

\noindent\textbf{(2) Hessian and Convexity:}
The Hessian of $E(w)$ is
\[
H = \nabla_w^2 E(w) = \Phi^\top \Phi.
\]
For any nonzero vector $z \in \mathbb{R}^M$,
\[
z^\top H z = z^\top \Phi^\top \Phi z = \|\Phi z\|^2 \ge 0,
\]
hence $H$ is positive semidefinite, implying $E(w)$ is convex.

\noindent If $\Phi$ has full column rank ($\mathrm{rank}(\Phi) = M$), then $\Phi^\top \Phi$ is positive definite, and
\[
z^\top H z = 0 \quad \Leftrightarrow \quad z = 0,
\]
so $E(w)$ is strictly convex.  
A strictly convex function has a unique minimizer, obtained by solving (1):
\[
w^* = (\Phi^\top \Phi)^{-1}\Phi^\top t.
\]

\noindent\textbf{(3) Non-uniqueness for Rank-Deficient $\Phi$:}
If $\Phi^\top \Phi$ is singular, there exist nonzero vectors $v$ such that $\Phi v = 0$.
For any particular solution $w_0$ satisfying (1), we have
\[
\Phi^\top \Phi (w_0 + v) = \Phi^\top \Phi w_0 + \Phi^\top \Phi v = \Phi^\top t,
\]
since $\Phi v = 0$.  
Thus, every vector $w = w_0 + v$, with $v \in \mathrm{Null}(\Phi)$, minimizes $E(w)$.
The minimal-norm solution among them is given by the Moore--Penrose pseudoinverse:
\[
w^* = \Phi^+ t.
\]

\noindent\textbf{(4) Conclusion:}
The cost $E(w)$ is convex for all $\Phi$, and strictly convex (hence uniquely minimized) iff $\Phi^\top \Phi$ is invertible.
\end{proof}
% ============================================================
\subsection{}
% ============================================================
\begin{theorem}[Unbiasedness of the OLS Estimator]
Assume the linear regression model
\[
t = \Phi w + \varepsilon,
\]
where \(\Phi\in\mathbb{R}^{N\times M}\) is the design matrix, \(w\in\mathbb{R}^M\) the true parameter vector, and the noise satisfies \(\mathbb{E}[\varepsilon]=0\) and \(\mathrm{Cov}(\varepsilon)=\sigma^2 I\).  
Assume further that \(\Phi^\top\Phi\) is invertible. Then the ordinary least squares estimator
\[
\hat{w} = (\Phi^\top\Phi)^{-1}\Phi^\top t
\]
is an unbiased estimator of \(w\), i.e.
\[
\mathbb{E}[\hat{w}] = w.
\]
\end{theorem}

\begin{proof}
By the model,
\[
t = \Phi w + \varepsilon.
\]
Substitute into the estimator:
\[
\hat{w} = (\Phi^\top\Phi)^{-1}\Phi^\top t
= (\Phi^\top\Phi)^{-1}\Phi^\top(\Phi w + \varepsilon).
\]
Distribute terms:
\[
\hat{w} = (\Phi^\top\Phi)^{-1}\Phi^\top\Phi\, w \;+\; (\Phi^\top\Phi)^{-1}\Phi^\top \varepsilon.
\]
Since \((\Phi^\top\Phi)^{-1}\Phi^\top\Phi = I_{M}\), this simplifies to
\[
\hat{w} = w + (\Phi^\top\Phi)^{-1}\Phi^\top \varepsilon.
\]
Take expectation using linearity and \(\mathbb{E}[\varepsilon]=0\):
\[
\mathbb{E}[\hat{w}] = \mathbb{E}\big[w + (\Phi^\top\Phi)^{-1}\Phi^\top \varepsilon\big]
= w + (\Phi^\top\Phi)^{-1}\Phi^\top \mathbb{E}[\varepsilon]
= w + (\Phi^\top\Phi)^{-1}\Phi^\top \,0
= w.
\]
Thus \(\hat{w}\) is unbiased.
\end{proof}
% ============================================================
\begin{corollary}
Under the same assumptions,
\[
\mathrm{Cov}(\hat{w}) = \sigma^2 (\Phi^\top\Phi)^{-1}.
\]
\end{corollary}

\begin{proof}
From $\hat{w} = w + (\Phi^\top\Phi)^{-1}\Phi^\top \varepsilon$ and $\mathrm{Cov}(\varepsilon)=\sigma^2 I$,
\[
\mathrm{Cov}(\hat{w}) = (\Phi^\top\Phi)^{-1}\Phi^\top \,\mathrm{Cov}(\varepsilon)\,\Phi(\Phi^\top\Phi)^{-1}
= \sigma^2 (\Phi^\top\Phi)^{-1}\Phi^\top\Phi(\Phi^\top\Phi)^{-1}
= \sigma^2 (\Phi^\top\Phi)^{-1}.
\]
\end{proof}

% ============================================================
\begin{theorem}[Covariance of the OLS Estimator]
Under the linear regression model
\[
t = \Phi w + \varepsilon, \qquad \mathbb{E}[\varepsilon]=0, \quad 
\mathrm{Cov}(\varepsilon)=\sigma^2 I,
\]
with $\Phi\in\mathbb{R}^{N\times M}$ of full column rank, the ordinary least squares estimator
\[
\hat{w} = (\Phi^\top\Phi)^{-1}\Phi^\top t
\]
has covariance matrix
\[
\mathrm{Cov}(\hat{w}) = \sigma^2 (\Phi^\top\Phi)^{-1}.
\]
\end{theorem}

\begin{proof}
From the model \(t = \Phi w + \varepsilon\),
\[
\hat{w} = (\Phi^\top\Phi)^{-1}\Phi^\top t
         = (\Phi^\top\Phi)^{-1}\Phi^\top(\Phi w + \varepsilon)
         = w + (\Phi^\top\Phi)^{-1}\Phi^\top \varepsilon.
\]
Subtract the expectation \( \mathbb{E}[\hat{w}] = w \) to get the deviation:
\[
\hat{w} - \mathbb{E}[\hat{w}] = (\Phi^\top\Phi)^{-1}\Phi^\top \varepsilon.
\]
Now compute the covariance:
\[
\begin{aligned}
\mathrm{Cov}(\hat{w})
&= \mathbb{E}\big[(\hat{w}-\mathbb{E}[\hat{w}])
  (\hat{w}-\mathbb{E}[\hat{w}])^\top\big] \\
&= \mathbb{E}\big[(\Phi^\top\Phi)^{-1}\Phi^\top 
     \varepsilon \varepsilon^\top
     \Phi(\Phi^\top\Phi)^{-1}\big].
\end{aligned}
\]
Using $\mathrm{Cov}(\varepsilon)=\sigma^2 I$ and the linearity of expectation:
\[
\mathrm{Cov}(\hat{w})
= (\Phi^\top\Phi)^{-1}\Phi^\top (\sigma^2 I) \Phi (\Phi^\top\Phi)^{-1}
= \sigma^2 (\Phi^\top\Phi)^{-1}\Phi^\top\Phi(\Phi^\top\Phi)^{-1}.
\]
Simplifying:
\[
\boxed{\mathrm{Cov}(\hat{w}) = \sigma^2 (\Phi^\top\Phi)^{-1}}.
\]
\end{proof}
% ============================================================
% ============================================================
\begin{theorem}[Gauss--Markov Theorem]
Consider the linear model
\[
t = \Phi w + \varepsilon,
\]
with $\Phi\in\mathbb{R}^{N\times M}$ of full column rank, $\mathbb{E}[\varepsilon]=0$, and $\mathrm{Cov}(\varepsilon)=\sigma^2 I$.  
Let $\hat{w}_{\mathrm{OLS}}=(\Phi^\top\Phi)^{-1}\Phi^\top t$ denote the ordinary least squares estimator.  
Then \(\hat{w}_{\mathrm{OLS}}\) is the \emph{Best Linear Unbiased Estimator (BLUE)}: for any other linear unbiased estimator of the form $\tilde w = C t$ (with constant matrix $C\in\mathbb{R}^{M\times N}$ such that $\mathbb{E}[\tilde w]=w$), we have
\[
\mathrm{Cov}(\tilde w) - \mathrm{Cov}(\hat{w}_{\mathrm{OLS}})
\; \succeq \; 0,
\]
i.e. the matrix difference is positive semidefinite. Equivalently, every componentwise variance of $\tilde w$ is at least that of $\hat{w}_{\mathrm{OLS}}$.
\end{theorem}

\begin{proof}
Let $\tilde w$ be any linear estimator of the form $\tilde w = C t$ for a fixed matrix $C\in\mathbb{R}^{M\times N}$.  
The unbiasedness condition $\mathbb{E}[\tilde w]=w$ requires
\[
\mathbb{E}[C t] = C \mathbb{E}[t] = C \Phi w = w \quad\text{for all }w,
\]
hence
\[
C\Phi = I_{M}. \tag{1}
\]

Write the OLS estimator as
\[
\hat{w} \equiv \hat{w}_{\mathrm{OLS}} = (\Phi^\top\Phi)^{-1}\Phi^\top t.
\]
Define the matrix difference
\[
A \;:=\; C - (\Phi^\top\Phi)^{-1}\Phi^\top.
\]
Using $(1)$ and the identity $\big((\Phi^\top\Phi)^{-1}\Phi^\top\big)\Phi = I_M$, we obtain
\[
A\Phi = C\Phi - (\Phi^\top\Phi)^{-1}\Phi^\top\Phi = I_M - I_M = 0.
\]
Thus
\[
A\Phi = 0 \qquad\Longrightarrow\qquad A\Phi w = 0 \quad\text{for all }w.
\]

Now express $\tilde w$ in terms of $\hat{w}$ and $A$:
\[
\tilde w = C t = \big((\Phi^\top\Phi)^{-1}\Phi^\top + A\big)t
= \hat{w} + A t.
\]
Subtracting expectations (and using $\mathbb{E}[\hat{w}]=\mathbb{E}[\tilde w]=w$) gives the zero-mean deviations
\[
\tilde w - w = (\hat{w} - w) + A\varepsilon,
\]
since $t=\Phi w + \varepsilon$ and $A\Phi w = 0$.

Compute the covariance matrices. Using $\mathrm{Cov}(\varepsilon)=\sigma^2 I$ and independence of deterministic matrices from $\varepsilon$,
\[
\begin{aligned}
\mathrm{Cov}(\tilde w)
&= \mathbb{E}\big[(\tilde w - w)(\tilde w - w)^\top\big] \\
&= \mathbb{E}\big[(\hat{w}-w + A\varepsilon)(\hat{w}-w + A\varepsilon)^\top\big] \\
&= \mathrm{Cov}(\hat{w}) + A\,\mathbb{E}[\varepsilon\varepsilon^\top]\,A^\top
    \;+\; \mathbb{E}\big[(\hat{w}-w)\varepsilon^\top\big]A^\top
    \;+\; A\,\mathbb{E}\big[\varepsilon(\hat{w}-w)^\top\big].
\end{aligned}
\]
But $\hat{w}-w = (\Phi^\top\Phi)^{-1}\Phi^\top\varepsilon$ is linear in $\varepsilon$, so
\[
\mathbb{E}\big[(\hat{w}-w)\varepsilon^\top\big]
= (\Phi^\top\Phi)^{-1}\Phi^\top \mathbb{E}[\varepsilon\varepsilon^\top]
= (\Phi^\top\Phi)^{-1}\Phi^\top (\sigma^2 I)
= \sigma^2 (\Phi^\top\Phi)^{-1}\Phi^\top.
\]
Since $A\Phi=0$, we have
\[
\mathbb{E}\big[(\hat{w}-w)\varepsilon^\top\big] A^\top
= \sigma^2 (\Phi^\top\Phi)^{-1}\Phi^\top A^\top
= \sigma^2 (\Phi^\top\Phi)^{-1} ( \Phi^\top A^\top )
= \sigma^2 (\Phi^\top\Phi)^{-1} (A\Phi)^\top
= 0.
\]
Similarly the other cross term $A\,\mathbb{E}[\varepsilon(\hat{w}-w)^\top]$ vanishes. Thus the covariance simplifies to
\[
\mathrm{Cov}(\tilde w)
= \mathrm{Cov}(\hat{w}) + A\,\mathbb{E}[\varepsilon\varepsilon^\top]\,A^\top
= \mathrm{Cov}(\hat{w}) + \sigma^2 A A^\top.
\]

Therefore
\[
\mathrm{Cov}(\tilde w) - \mathrm{Cov}(\hat{w}) = \sigma^2 A A^\top.
\]
But $\sigma^2 A A^\top$ is positive semidefinite (for any $\sigma^2\ge 0$ and any matrix $A$), so
\[
\mathrm{Cov}(\tilde w) - \mathrm{Cov}(\hat{w}) \succeq 0,
\]
which proves that $\hat{w}$ has the smallest covariance matrix among all linear unbiased estimators. This completes the proof.
\end{proof}

% ============================================================
% ============================================================
\begin{theorem}[Orthogonality of Residuals]
Let $\Phi\in\mathbb{R}^{N\times M}$ be the design matrix and $t\in\mathbb{R}^N$ the observed targets.
Let $\hat w$ be any solution of the normal equations
\[
\Phi^\top \Phi \,\hat w \;=\; \Phi^\top t.
\]
Define the residual vector $r := t - \Phi \hat w$. Then
\[
\Phi^\top r \;=\; 0,
\]
i.e. $r$ is orthogonal to every column of $\Phi$ (equivalently $r$ is orthogonal to $\operatorname{col}(\Phi)$).
\end{theorem}

\begin{proof}
Starting from the normal equations,
\[
\Phi^\top \Phi \,\hat w \;=\; \Phi^\top t.
\]
Rearrange terms to move $\Phi^\top\Phi\hat w$ to the right-hand side:
\[
\Phi^\top t - \Phi^\top \Phi \,\hat w \;=\; 0.
\]
Factor $\Phi^\top$:
\[
\Phi^\top (t - \Phi \hat w) \;=\; 0.
\]
But $t - \Phi\hat w$ is exactly the residual vector $r$, hence
\[
\Phi^\top r \;=\; 0.
\]
This shows each column of $\Phi$ has zero inner product with $r$, i.e. $r\perp \operatorname{col}(\Phi)$.
\end{proof}

\begin{corollary}[Hat Matrix and Residual Projection]
If $\Phi$ has full column rank and $\hat w = (\Phi^\top\Phi)^{-1}\Phi^\top t$, define the hat (projection) matrix
\[
P := \Phi(\Phi^\top\Phi)^{-1}\Phi^\top.
\]
Then the fitted values are $\hat t = P t$ and the residual satisfies
\[
r = (I - P) t,
\]
with $P^2 = P$ and $P^\top = P$. Consequently $(I-P)$ is the orthogonal projector onto $\operatorname{col}(\Phi)^\perp$, and $r$ is the orthogonal projection of $t$ onto that complement.
\end{corollary}

\begin{proof}
Using $\hat w = (\Phi^\top\Phi)^{-1}\Phi^\top t$ gives $\hat t = \Phi\hat w = \Phi(\Phi^\top\Phi)^{-1}\Phi^\top t = Pt$, so $r = t - \hat t = (I-P)t$.
The identities $P^2=P$ and $P^\top=P$ follow from straightforward algebra:
\[
P^2 = \Phi(\Phi^\top\Phi)^{-1}\underbrace{\Phi^\top\Phi}_{=}\;(\Phi^\top\Phi)^{-1}\Phi^\top = P,
\qquad
P^\top = \big(\Phi(\Phi^\top\Phi)^{-1}\Phi^\top\big)^\top = \Phi(\Phi^\top\Phi)^{-1}\Phi^\top = P.
\]
Thus $P$ is an orthogonal projector onto $\operatorname{col}(\Phi)$ and $(I-P)$ projects orthogonally onto its complement, so $r$ lies in $\operatorname{col}(\Phi)^\perp$.
\end{proof}
% ============================================================
% ============================================================
\section*{Bayesian Linear Regression: Prior on $w$ and Predictive Distribution}

\subsection*{Bayesian Formulation}

In Bayesian linear regression we treat the parameter vector $w$ as a random variable and place a prior distribution on it.  
The generative model is:
\[
t = \Phi w + \varepsilon, 
\qquad 
\varepsilon \sim \mathcal{N}(0, \beta^{-1} I_N),
\]
where $\beta$ is the noise precision.

\subsection*{Prior Distribution on $w$}
We choose a zero-mean isotropic Gaussian prior:
\[
p(w) = \mathcal{N}(w \mid 0, \alpha^{-1} I_M),
\]
where $\alpha$ is the prior precision.  
This encodes the belief that large weights are unlikely (acts as a regularizer).

\subsection*{Likelihood}
Conditioned on $w$, the likelihood of the data is:
\[
p(t \mid \Phi, w, \beta)
= \mathcal{N}(t \mid \Phi w, \beta^{-1} I_N).
\]

\subsection*{Posterior Distribution of $w$}

By Bayes' theorem:
\[
p(w \mid t, \Phi)
\propto p(t \mid \Phi, w, \beta)\, p(w).
\]

Because both prior and likelihood are Gaussian, the posterior is also Gaussian:
\[
p(w \mid t, \Phi) 
= \mathcal{N}(w \mid m_N, S_N),
\]
with posterior precision and covariance given by:
\[
S_N^{-1} = \alpha I_M + \beta \Phi^\top \Phi,
\qquad 
S_N = (\alpha I_M + \beta \Phi^\top \Phi)^{-1},
\]
and the posterior mean:
\[
m_N = \beta S_N \Phi^\top t.
\]

\subsection*{Interpretation}
\begin{itemize}
    \item $m_N$ is the Bayes estimate of $w$ (posterior mean).
    \item $S_N$ quantifies uncertainty in the weight estimates.
    \item As $\alpha \to 0$ (weak prior),  
    \[
    m_N \to (\Phi^\top \Phi)^{-1}\Phi^\top t,
    \]
    recovering the ordinary least squares solution.
\end{itemize}

\subsection*{Predictive Distribution}

For a new input $x_*$ with feature vector $\phi_* = \phi(x_*)$, the predictive distribution integrates over the posterior uncertainty in $w$:
\[
p(t_* \mid x_*, t, \Phi)
= \int p(t_* \mid x_*, w, \beta)\, p(w \mid t, \Phi)\, dw.
\]

The integrand is a product of two Gaussians, so the predictive distribution is Gaussian:
\[
p(t_* \mid x_*, t, \Phi)
= \mathcal{N}\big(t_* \mid m_N^\top \phi_*,\ 
\beta^{-1} + \phi_*^\top S_N \phi_* \big).
\]

\subsection*{Predictive Mean and Variance}

\paragraph{Predictive Mean:}
\[
\mathbb{E}[t_* \mid x_*, t, \Phi]
= m_N^\top \phi_*.
\]

\paragraph{Predictive Variance:}
\[
\mathrm{Var}(t_* \mid x_*, t, \Phi)
= \underbrace{\beta^{-1}}_{\text{noise variance}}
\;+\;
\underbrace{\phi_*^\top S_N \phi_*}_{\text{model uncertainty}}.
\]

Thus the predictive variance decomposes into:
\begin{itemize}
    \item aleatoric noise (irreducible), and
    \item epistemic uncertainty (reduced with more data).
\end{itemize}
% ============================================================
% ============================================================
\section*{Likelihood Derivation (Gaussian Noise) and MLEs}

\subsection*{1. Single-observation likelihood}
Assume the data generation model for a single observation:
\[
t_n = w^\top \phi(x_n) + \varepsilon_n,\qquad \varepsilon_n \sim \mathcal{N}(0,\beta^{-1}).
\]
Then the conditional density (likelihood) for $t_n$ given $w$ is
\[
p(t_n \mid x_n, w, \beta) = \mathcal{N}\big(t_n \mid w^\top\phi(x_n),\ \beta^{-1}\big)
= \sqrt{\frac{\beta}{2\pi}}\,
\exp\!\Big(-\tfrac{\beta}{2}\big(t_n - w^\top\phi(x_n)\big)^2\Big).
\]

\subsection*{2. Joint likelihood for the dataset}
Assuming i.i.d.\ noise, the joint likelihood for all $N$ observations is the product
\[
p(t \mid \Phi, w, \beta)
= \prod_{n=1}^N p(t_n \mid x_n, w, \beta)
= \left(\frac{\beta}{2\pi}\right)^{\!N/2}
\exp\!\Big(-\tfrac{\beta}{2}\sum_{n=1}^N (t_n - w^\top\phi(x_n))^2\Big).
\]
Using matrix notation with $\Phi\in\mathbb{R}^{N\times M}$ and $t\in\mathbb{R}^N$:
\[
p(t \mid \Phi, w, \beta)
= \left(\frac{\beta}{2\pi}\right)^{\!N/2}
\exp\!\Big(-\tfrac{\beta}{2}\|t - \Phi w\|^2\Big).
\]

\subsection*{3. Log-likelihood}
The log-likelihood (more convenient for optimization) is
\[
\ell(w,\beta) := \log p(t\mid \Phi, w, \beta)
= \frac{N}{2}\log\beta - \frac{N}{2}\log(2\pi) - \frac{\beta}{2}\|t - \Phi w\|^2.
\]
Dropping constants independent of the parameters when optimizing:
\[
\ell(w,\beta) = \frac{N}{2}\log\beta - \frac{\beta}{2}\|t - \Phi w\|^2 + \text{const}.
\]

\subsection*{4. MLE for $w$ (given $\beta$)}
Take gradient of the log-likelihood w.r.t.\ $w$:
\[
\nabla_w \ell(w,\beta)
= -\frac{\beta}{2}\cdot 2\,(-\Phi^\top)(t - \Phi w)
= \beta \Phi^\top (t - \Phi w).
\]
Set to zero for critical point:
\[
\Phi^\top (t - \Phi w) = 0
\quad\Rightarrow\quad
\Phi^\top \Phi\, w = \Phi^\top t.
\]
If $\Phi^\top\Phi$ is invertible, the MLE of $w$ is
\[
\boxed{\,\hat w_{\mathrm{MLE}} = (\Phi^\top\Phi)^{-1}\Phi^\top t\,}
\]
which is the ordinary least squares solution. Thus MLE \(=\) least squares under Gaussian noise.

\subsection*{5. MLE for noise precision $\beta$ (given $w$)}
Differentiate $\ell$ w.r.t.\ $\beta$:
\[
\frac{\partial \ell}{\partial \beta}
= \frac{N}{2\beta} - \frac{1}{2}\|t - \Phi w\|^2.
\]
Set equal to zero:
\[
\frac{N}{2\beta} = \frac{1}{2}\|t - \Phi w\|^2
\quad\Rightarrow\quad
\hat\beta_{\mathrm{MLE}} = \frac{N}{\|t - \Phi w\|^2}.
\]
If we substitute $w=\hat w_{\mathrm{MLE}}$ we get the MLE for $\beta$:
\[
\boxed{\,\hat\beta_{\mathrm{MLE}} = \frac{N}{\|t - \Phi \hat w_{\mathrm{MLE}}\|^2}\, }.
\]
Equivalently, the MLE for noise variance $\sigma^2=\beta^{-1}$ is
\[
\hat\sigma^2_{\mathrm{MLE}} = \frac{1}{N}\|t - \Phi \hat w_{\mathrm{MLE}}\|^2.
\]
(For an unbiased estimator of $\sigma^2$ divide by $N-M$ instead of $N$.)

\subsection*{6. Negative log-likelihood and connection to MAP}
The negative log-likelihood (up to additive constant) is
\[
-\ell(w,\beta) \propto \frac{\beta}{2}\|t - \Phi w\|^2 - \frac{N}{2}\log\beta.
\]
When combining with a Gaussian prior $p(w)\propto\exp(-\tfrac{\alpha}{2}\|w\|^2)$,
the negative log-posterior (up to constants) becomes
\[
-\log p(w\mid t) \propto \frac{\beta}{2}\|t - \Phi w\|^2 + \frac{\alpha}{2}\|w\|^2,
\]
whose minimizer yields the MAP estimator. Dividing through by $\beta$ and setting $\lambda=\alpha/\beta$ gives the familiar ridge form:
\[
\hat w_{\mathrm{MAP}} = (\Phi^\top\Phi + \lambda I)^{-1}\Phi^\top t.
\]

% ============================================================

% ============================================================
\section*{Derivation of the Posterior with a Gaussian Prior (Completing the Square)}

Assume the Gaussian likelihood and Gaussian prior:
\[
p(t\mid w) \propto \exp\!\Big(-\tfrac{\beta}{2}\|t-\Phi w\|^2\Big), 
\qquad
p(w) \propto \exp\!\Big(-\tfrac{\alpha}{2}\|w\|^2\Big).
\]
Posterior (unnormalized) by Bayes' rule:
\[
p(w\mid t) \propto p(t\mid w)\,p(w)
\propto \exp\!\Big(-\tfrac{\beta}{2}\|t-\Phi w\|^2 - \tfrac{\alpha}{2}\|w\|^2\Big).
\]

\paragraph{Expand the exponents (quadratic form in $w$).}
\[
\begin{aligned}
&\quad \; \tfrac{\beta}{2}\|t-\Phi w\|^2 + \tfrac{\alpha}{2}\|w\|^2 \\
&= \tfrac{\beta}{2}\big(t^\top t - 2t^\top\Phi w + w^\top\Phi^\top\Phi w\big)
   + \tfrac{\alpha}{2} w^\top w \\
&= \tfrac{1}{2}\, w^\top(\beta\Phi^\top\Phi + \alpha I)\,w \;-\; \beta t^\top\Phi w 
   + \tfrac{\beta}{2} t^\top t.
\end{aligned}
\]

\paragraph{Group terms in $w$ and complete the square.}
Write the quadratic form as
\[
\tfrac{1}{2}\, w^\top A\, w - b^\top w + \text{const},
\quad\text{where } A = \beta\Phi^\top\Phi + \alpha I,\quad b = \beta \Phi^\top t.
\]
Complete the square:
\[
\tfrac{1}{2} w^\top A w - b^\top w
= \tfrac{1}{2}(w - A^{-1}b)^\top A (w - A^{-1}b) - \tfrac{1}{2} b^\top A^{-1} b.
\]
Thus the unnormalized posterior becomes
\[
p(w\mid t) \propto \exp\!\Big(-\tfrac{1}{2}(w - A^{-1}b)^\top A (w - A^{-1}b)\Big)
\cdot \exp\!\Big(\tfrac{1}{2} b^\top A^{-1} b - \tfrac{\beta}{2} t^\top t\Big).
\]
The second exponential is independent of \(w\) and becomes part of the normalizing constant.

\paragraph{Identify posterior covariance and mean.}
Hence the posterior is Gaussian with precision \(A\) and covariance \(S_N = A^{-1}\):
\[
S_N = (\beta\Phi^\top\Phi + \alpha I)^{-1},
\]
and posterior mean
\[
m_N = A^{-1} b = (\beta\Phi^\top\Phi + \alpha I)^{-1} (\beta \Phi^\top t).
\]

\paragraph{Simplify using $\lambda=\alpha/\beta$.}
Dividing numerator and denominator by \(\beta\) gives the more familiar form:
\[
S_N = \beta^{-1}(\Phi^\top\Phi + \lambda I)^{-1},\qquad
m_N = (\Phi^\top\Phi + \lambda I)^{-1}\Phi^\top t,
\]
where \(\lambda = \alpha/\beta\). Note that \(m_N\) equals the ridge/MAP estimator and \(S_N\) quantifies posterior uncertainty.
% ============================================================



    
\end{multicols}
\end{document}