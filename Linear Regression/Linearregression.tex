\documentclass[10pt, landscape]{report} % landscape for wider columns
\usepackage[a4paper,margin=0.2in]{geometry} % tight margins for dense layout
\usepackage{multicol}                     % multiple columns
\usepackage{amsmath,amssymb,mathtools,bm} % math packages
\usepackage{physics}                      % handy for derivatives, norms, etc.
\usepackage{booktabs,tabularx}            % tables
\usepackage{tikz,pgfplots}                % diagrams or plots (optional)
\usepackage{microtype}                    % better text spacing
\usepackage{enumitem}                     % compact item lists
\usepackage{fancyhdr}                     % custom header/footer
\usepackage{hyperref}                     % clickable refs
\usepackage{titlesec}                     % section spacing control
\usepackage{siunitx}                      % nice number formatting
\usepackage{xcolor}                       % color for highlights
\usepackage{tcolorbox}                    % boxed formulas
\usepackage{parskip}                      % spacing between paragraphs
\usepackage{amsthm}

% -------------------------------------------------------------
% PAGE & TEXT DENSITY SETTINGS
% -------------------------------------------------------------
\setlength{\parindent}{0pt}
\setlength{\parskip}{0pt}
\setlist[itemize]{noitemsep, topsep=1pt, left=1mm}
\setlist[enumerate]{noitemsep, topsep=1pt, left=1mm}
\setlength{\columnsep}{0.2in} % space between columns
\renewcommand{\baselinestretch}{0.92} % slightly denser lines
\renewcommand{\familydefault}{\sfdefault} % clean sans-serif font for readability

% Adjust section titles for compactness
\titleformat{\section}{\bfseries\large\color{blue!70!black}}{}{0em}{}
\titleformat{\subsection}{\bfseries\normalsize\color{teal!70!black}}{}{0em}{}
\titlespacing*{\section}{0pt}{1ex}{0.5ex}
\titlespacing*{\subsection}{0pt}{0.5ex}{0.25ex}

% -------------------------------------------------------------
% HEADER & FOOTER
% -------------------------------------------------------------
\pagestyle{fancy}
\fancyhf{}
\lhead{\textbf{Machine Learning Cheat Sheet — Open Book}}
\rhead{\today}
\cfoot{\thepage}

% -------------------------------------------------------------
% MATH SHORTCUTS
% -------------------------------------------------------------
\newcommand{\R}{\mathbb{R}}
\newcommand{\E}{\mathbb{E}}
\newcommand{\Var}{\mathrm{Var}}
\newcommand{\Cov}{\mathrm{Cov}}
\newcommand{\KL}{\mathrm{D_{KL}}}
\newcommand{\N}{\mathcal{N}}
\newcommand{\softmax}{\mathrm{softmax}}
\newcommand{\sigmoid}{\sigma}
\newcommand{\argmin}{\operatorname*{arg\,min}}
\newcommand{\argmax}{\operatorname*{arg\,max}}
\newcommand{\ind}{\mathbf{1}}

% -------------------------------------------------------------
% BOXED FORMULAS (optional)
% -------------------------------------------------------------
\tcbset{
  colback=white,
  colframe=blue!50!black,
  boxrule=0.4pt,
  arc=2pt,
  left=3pt,
  right=3pt,
  top=2pt,
  bottom=2pt,
  fonttitle=\bfseries\color{blue!60!black}, % <- use fonttitle directly
  coltitle=blue!60!black                    % <- sets title text color safely
}


\newtcolorbox{formula}[1][]{
  colback=blue!2!white,
  colframe=blue!50!black,
  title=#1,
  boxrule=0.3pt,
  sharp corners,
  fontupper=\footnotesize
}

% -------------------------------------------------------------
% CUSTOM SMALL FONTS FOR CHEAT SHEET
% -------------------------------------------------------------
\newcommand{\tinyfont}{\fontsize{7.5pt}{8.5pt}\selectfont}
\newcommand{\smallfont}{\fontsize{8.5pt}{9.5pt}\selectfont}

\theoremstyle{plain}
\newtheorem{theorem}{Theorem}[section]         % Theorems numbered within sections
\newtheorem{lemma}[theorem]{Lemma}             % Lemmas share numbering with theorems
\newtheorem{proposition}[theorem]{Proposition} % Propositions share numbering
\newtheorem{corollary}[theorem]{Corollary}     % Corollaries share numbering

\theoremstyle{definition}
\newtheorem{definition}[theorem]{Definition}   % Definitions have upright text
\newtheorem{example}[theorem]{Example}         % Examples same numbering style

\theoremstyle{remark}
\newtheorem{remark}[theorem]{Remark}           % Remarks in italic header but normal text
\newtheorem{note}[theorem]{Note}



\begin{document}
% ============================================================
\chapter{Linear Regression}
\begin{multicols}{2}
\section*{Ordinary Least Squares}
\subsection{}
\begin{theorem}[Uniqueness of the Least Squares Solution]
Let $\Phi \in \mathbb{R}^{N \times M}$ denote the design matrix and $t \in \mathbb{R}^N$ the target vector.
Consider the least squares cost function
\[
E(w) = \frac{1}{2}\|t - \Phi w\|^2.
\]
Then:
\begin{enumerate}[label=(\roman*)]
    \item The function $E(w)$ is convex in $w$.
    \item If $\Phi^\top \Phi$ is invertible (i.e., $\mathrm{rank}(\Phi) = M$), then $E(w)$ is \emph{strictly convex} and admits a unique minimizer
    \[
    w^* = (\Phi^\top \Phi)^{-1}\Phi^\top t.
    \]
    \item If $\Phi^\top \Phi$ is singular, the minimizer is not unique; all minimizers are of the form
    \[
    w = w_0 + v, \qquad v \in \mathrm{Null}(\Phi),
    \]
    where $w_0$ is any particular solution to the normal equations
    $\Phi^\top \Phi w = \Phi^\top t$.
\end{enumerate}
\end{theorem}

\begin{proof}
We begin by expanding the objective:
\[
E(w) = \frac{1}{2}(t - \Phi w)^\top (t - \Phi w)
      = \frac{1}{2}(t^\top t - 2t^\top \Phi w + w^\top \Phi^\top \Phi w).
\]

\noindent\textbf{(1) Gradient and Stationary Point:}
The gradient of $E(w)$ with respect to $w$ is
\[
\nabla_w E(w) = -\Phi^\top t + \Phi^\top \Phi w.
\]
Setting $\nabla_w E(w) = 0$ yields the \emph{normal equations}
\[
\Phi^\top \Phi w = \Phi^\top t. \tag{1}
\]

\noindent\textbf{(2) Hessian and Convexity:}
The Hessian of $E(w)$ is
\[
H = \nabla_w^2 E(w) = \Phi^\top \Phi.
\]
For any nonzero vector $z \in \mathbb{R}^M$,
\[
z^\top H z = z^\top \Phi^\top \Phi z = \|\Phi z\|^2 \ge 0,
\]
hence $H$ is positive semidefinite, implying $E(w)$ is convex.

\noindent If $\Phi$ has full column rank ($\mathrm{rank}(\Phi) = M$), then $\Phi^\top \Phi$ is positive definite, and
\[
z^\top H z = 0 \quad \Leftrightarrow \quad z = 0,
\]
so $E(w)$ is strictly convex.  
A strictly convex function has a unique minimizer, obtained by solving (1):
\[
w^* = (\Phi^\top \Phi)^{-1}\Phi^\top t.
\]

\noindent\textbf{(3) Non-uniqueness for Rank-Deficient $\Phi$:}
If $\Phi^\top \Phi$ is singular, there exist nonzero vectors $v$ such that $\Phi v = 0$.
For any particular solution $w_0$ satisfying (1), we have
\[
\Phi^\top \Phi (w_0 + v) = \Phi^\top \Phi w_0 + \Phi^\top \Phi v = \Phi^\top t,
\]
since $\Phi v = 0$.  
Thus, every vector $w = w_0 + v$, with $v \in \mathrm{Null}(\Phi)$, minimizes $E(w)$.
The minimal-norm solution among them is given by the Moore--Penrose pseudoinverse:
\[
w^* = \Phi^+ t.
\]

\noindent\textbf{(4) Conclusion:}
The cost $E(w)$ is convex for all $\Phi$, and strictly convex (hence uniquely minimized) iff $\Phi^\top \Phi$ is invertible.
\end{proof}
% ============================================================
\subsection{}
% ============================================================
\begin{theorem}[Unbiasedness of the OLS Estimator]
Assume the linear regression model
\[
t = \Phi w + \varepsilon,
\]
where \(\Phi\in\mathbb{R}^{N\times M}\) is the design matrix, \(w\in\mathbb{R}^M\) the true parameter vector, and the noise satisfies \(\mathbb{E}[\varepsilon]=0\) and \(\mathrm{Cov}(\varepsilon)=\sigma^2 I\).  
Assume further that \(\Phi^\top\Phi\) is invertible. Then the ordinary least squares estimator
\[
\hat{w} = (\Phi^\top\Phi)^{-1}\Phi^\top t
\]
is an unbiased estimator of \(w\), i.e.
\[
\mathbb{E}[\hat{w}] = w.
\]
\end{theorem}

\begin{proof}
By the model,
\[
t = \Phi w + \varepsilon.
\]
Substitute into the estimator:
\[
\hat{w} = (\Phi^\top\Phi)^{-1}\Phi^\top t
= (\Phi^\top\Phi)^{-1}\Phi^\top(\Phi w + \varepsilon).
\]
Distribute terms:
\[
\hat{w} = (\Phi^\top\Phi)^{-1}\Phi^\top\Phi\, w \;+\; (\Phi^\top\Phi)^{-1}\Phi^\top \varepsilon.
\]
Since \((\Phi^\top\Phi)^{-1}\Phi^\top\Phi = I_{M}\), this simplifies to
\[
\hat{w} = w + (\Phi^\top\Phi)^{-1}\Phi^\top \varepsilon.
\]
Take expectation using linearity and \(\mathbb{E}[\varepsilon]=0\):
\[
\mathbb{E}[\hat{w}] = \mathbb{E}\big[w + (\Phi^\top\Phi)^{-1}\Phi^\top \varepsilon\big]
= w + (\Phi^\top\Phi)^{-1}\Phi^\top \mathbb{E}[\varepsilon]
= w + (\Phi^\top\Phi)^{-1}\Phi^\top \,0
= w.
\]
Thus \(\hat{w}\) is unbiased.
\end{proof}
% ============================================================
\begin{corollary}
Under the same assumptions,
\[
\mathrm{Cov}(\hat{w}) = \sigma^2 (\Phi^\top\Phi)^{-1}.
\]
\end{corollary}

\begin{proof}
From $\hat{w} = w + (\Phi^\top\Phi)^{-1}\Phi^\top \varepsilon$ and $\mathrm{Cov}(\varepsilon)=\sigma^2 I$,
\[
\mathrm{Cov}(\hat{w}) = (\Phi^\top\Phi)^{-1}\Phi^\top \,\mathrm{Cov}(\varepsilon)\,\Phi(\Phi^\top\Phi)^{-1}
= \sigma^2 (\Phi^\top\Phi)^{-1}\Phi^\top\Phi(\Phi^\top\Phi)^{-1}
= \sigma^2 (\Phi^\top\Phi)^{-1}.
\]
\end{proof}

% ============================================================
\begin{theorem}[Covariance of the OLS Estimator]
Under the linear regression model
\[
t = \Phi w + \varepsilon, \qquad \mathbb{E}[\varepsilon]=0, \quad 
\mathrm{Cov}(\varepsilon)=\sigma^2 I,
\]
with $\Phi\in\mathbb{R}^{N\times M}$ of full column rank, the ordinary least squares estimator
\[
\hat{w} = (\Phi^\top\Phi)^{-1}\Phi^\top t
\]
has covariance matrix
\[
\mathrm{Cov}(\hat{w}) = \sigma^2 (\Phi^\top\Phi)^{-1}.
\]
\end{theorem}

\begin{proof}
From the model \(t = \Phi w + \varepsilon\),
\[
\hat{w} = (\Phi^\top\Phi)^{-1}\Phi^\top t
         = (\Phi^\top\Phi)^{-1}\Phi^\top(\Phi w + \varepsilon)
         = w + (\Phi^\top\Phi)^{-1}\Phi^\top \varepsilon.
\]
Subtract the expectation \( \mathbb{E}[\hat{w}] = w \) to get the deviation:
\[
\hat{w} - \mathbb{E}[\hat{w}] = (\Phi^\top\Phi)^{-1}\Phi^\top \varepsilon.
\]
Now compute the covariance:
\[
\begin{aligned}
\mathrm{Cov}(\hat{w})
&= \mathbb{E}\big[(\hat{w}-\mathbb{E}[\hat{w}])
  (\hat{w}-\mathbb{E}[\hat{w}])^\top\big] \\
&= \mathbb{E}\big[(\Phi^\top\Phi)^{-1}\Phi^\top 
     \varepsilon \varepsilon^\top
     \Phi(\Phi^\top\Phi)^{-1}\big].
\end{aligned}
\]
Using $\mathrm{Cov}(\varepsilon)=\sigma^2 I$ and the linearity of expectation:
\[
\mathrm{Cov}(\hat{w})
= (\Phi^\top\Phi)^{-1}\Phi^\top (\sigma^2 I) \Phi (\Phi^\top\Phi)^{-1}
= \sigma^2 (\Phi^\top\Phi)^{-1}\Phi^\top\Phi(\Phi^\top\Phi)^{-1}.
\]
Simplifying:
\[
\boxed{\mathrm{Cov}(\hat{w}) = \sigma^2 (\Phi^\top\Phi)^{-1}}.
\]
\end{proof}
% ============================================================
% ============================================================
\begin{theorem}[Gauss--Markov Theorem]
Consider the linear model
\[
t = \Phi w + \varepsilon,
\]
with $\Phi\in\mathbb{R}^{N\times M}$ of full column rank, $\mathbb{E}[\varepsilon]=0$, and $\mathrm{Cov}(\varepsilon)=\sigma^2 I$.  
Let $\hat{w}_{\mathrm{OLS}}=(\Phi^\top\Phi)^{-1}\Phi^\top t$ denote the ordinary least squares estimator.  
Then \(\hat{w}_{\mathrm{OLS}}\) is the \emph{Best Linear Unbiased Estimator (BLUE)}: for any other linear unbiased estimator of the form $\tilde w = C t$ (with constant matrix $C\in\mathbb{R}^{M\times N}$ such that $\mathbb{E}[\tilde w]=w$), we have
\[
\mathrm{Cov}(\tilde w) - \mathrm{Cov}(\hat{w}_{\mathrm{OLS}})
\; \succeq \; 0,
\]
i.e. the matrix difference is positive semidefinite. Equivalently, every componentwise variance of $\tilde w$ is at least that of $\hat{w}_{\mathrm{OLS}}$.
\end{theorem}

\begin{proof}
Let $\tilde w$ be any linear estimator of the form $\tilde w = C t$ for a fixed matrix $C\in\mathbb{R}^{M\times N}$.  
The unbiasedness condition $\mathbb{E}[\tilde w]=w$ requires
\[
\mathbb{E}[C t] = C \mathbb{E}[t] = C \Phi w = w \quad\text{for all }w,
\]
hence
\[
C\Phi = I_{M}. \tag{1}
\]

Write the OLS estimator as
\[
\hat{w} \equiv \hat{w}_{\mathrm{OLS}} = (\Phi^\top\Phi)^{-1}\Phi^\top t.
\]
Define the matrix difference
\[
A \;:=\; C - (\Phi^\top\Phi)^{-1}\Phi^\top.
\]
Using $(1)$ and the identity $\big((\Phi^\top\Phi)^{-1}\Phi^\top\big)\Phi = I_M$, we obtain
\[
A\Phi = C\Phi - (\Phi^\top\Phi)^{-1}\Phi^\top\Phi = I_M - I_M = 0.
\]
Thus
\[
A\Phi = 0 \qquad\Longrightarrow\qquad A\Phi w = 0 \quad\text{for all }w.
\]

Now express $\tilde w$ in terms of $\hat{w}$ and $A$:
\[
\tilde w = C t = \big((\Phi^\top\Phi)^{-1}\Phi^\top + A\big)t
= \hat{w} + A t.
\]
Subtracting expectations (and using $\mathbb{E}[\hat{w}]=\mathbb{E}[\tilde w]=w$) gives the zero-mean deviations
\[
\tilde w - w = (\hat{w} - w) + A\varepsilon,
\]
since $t=\Phi w + \varepsilon$ and $A\Phi w = 0$.

Compute the covariance matrices. Using $\mathrm{Cov}(\varepsilon)=\sigma^2 I$ and independence of deterministic matrices from $\varepsilon$,
\[
\begin{aligned}
\mathrm{Cov}(\tilde w)
&= \mathbb{E}\big[(\tilde w - w)(\tilde w - w)^\top\big] \\
&= \mathbb{E}\big[(\hat{w}-w + A\varepsilon)(\hat{w}-w + A\varepsilon)^\top\big] \\
&= \mathrm{Cov}(\hat{w}) + A\,\mathbb{E}[\varepsilon\varepsilon^\top]\,A^\top
    \;+\; \mathbb{E}\big[(\hat{w}-w)\varepsilon^\top\big]A^\top
    \;+\; A\,\mathbb{E}\big[\varepsilon(\hat{w}-w)^\top\big].
\end{aligned}
\]
But $\hat{w}-w = (\Phi^\top\Phi)^{-1}\Phi^\top\varepsilon$ is linear in $\varepsilon$, so
\[
\mathbb{E}\big[(\hat{w}-w)\varepsilon^\top\big]
= (\Phi^\top\Phi)^{-1}\Phi^\top \mathbb{E}[\varepsilon\varepsilon^\top]
= (\Phi^\top\Phi)^{-1}\Phi^\top (\sigma^2 I)
= \sigma^2 (\Phi^\top\Phi)^{-1}\Phi^\top.
\]
Since $A\Phi=0$, we have
\[
\mathbb{E}\big[(\hat{w}-w)\varepsilon^\top\big] A^\top
= \sigma^2 (\Phi^\top\Phi)^{-1}\Phi^\top A^\top
= \sigma^2 (\Phi^\top\Phi)^{-1} ( \Phi^\top A^\top )
= \sigma^2 (\Phi^\top\Phi)^{-1} (A\Phi)^\top
= 0.
\]
Similarly the other cross term $A\,\mathbb{E}[\varepsilon(\hat{w}-w)^\top]$ vanishes. Thus the covariance simplifies to
\[
\mathrm{Cov}(\tilde w)
= \mathrm{Cov}(\hat{w}) + A\,\mathbb{E}[\varepsilon\varepsilon^\top]\,A^\top
= \mathrm{Cov}(\hat{w}) + \sigma^2 A A^\top.
\]

Therefore
\[
\mathrm{Cov}(\tilde w) - \mathrm{Cov}(\hat{w}) = \sigma^2 A A^\top.
\]
But $\sigma^2 A A^\top$ is positive semidefinite (for any $\sigma^2\ge 0$ and any matrix $A$), so
\[
\mathrm{Cov}(\tilde w) - \mathrm{Cov}(\hat{w}) \succeq 0,
\]
which proves that $\hat{w}$ has the smallest covariance matrix among all linear unbiased estimators. This completes the proof.
\end{proof}

% ============================================================
% ============================================================
\begin{theorem}[Orthogonality of Residuals]
Let $\Phi\in\mathbb{R}^{N\times M}$ be the design matrix and $t\in\mathbb{R}^N$ the observed targets.
Let $\hat w$ be any solution of the normal equations
\[
\Phi^\top \Phi \,\hat w \;=\; \Phi^\top t.
\]
Define the residual vector $r := t - \Phi \hat w$. Then
\[
\Phi^\top r \;=\; 0,
\]
i.e. $r$ is orthogonal to every column of $\Phi$ (equivalently $r$ is orthogonal to $\operatorname{col}(\Phi)$).
\end{theorem}

\begin{proof}
Starting from the normal equations,
\[
\Phi^\top \Phi \,\hat w \;=\; \Phi^\top t.
\]
Rearrange terms to move $\Phi^\top\Phi\hat w$ to the right-hand side:
\[
\Phi^\top t - \Phi^\top \Phi \,\hat w \;=\; 0.
\]
Factor $\Phi^\top$:
\[
\Phi^\top (t - \Phi \hat w) \;=\; 0.
\]
But $t - \Phi\hat w$ is exactly the residual vector $r$, hence
\[
\Phi^\top r \;=\; 0.
\]
This shows each column of $\Phi$ has zero inner product with $r$, i.e. $r\perp \operatorname{col}(\Phi)$.
\end{proof}

\begin{corollary}[Hat Matrix and Residual Projection]
If $\Phi$ has full column rank and $\hat w = (\Phi^\top\Phi)^{-1}\Phi^\top t$, define the hat (projection) matrix
\[
P := \Phi(\Phi^\top\Phi)^{-1}\Phi^\top.
\]
Then the fitted values are $\hat t = P t$ and the residual satisfies
\[
r = (I - P) t,
\]
with $P^2 = P$ and $P^\top = P$. Consequently $(I-P)$ is the orthogonal projector onto $\operatorname{col}(\Phi)^\perp$, and $r$ is the orthogonal projection of $t$ onto that complement.
\end{corollary}

\begin{proof}
Using $\hat w = (\Phi^\top\Phi)^{-1}\Phi^\top t$ gives $\hat t = \Phi\hat w = \Phi(\Phi^\top\Phi)^{-1}\Phi^\top t = Pt$, so $r = t - \hat t = (I-P)t$.
The identities $P^2=P$ and $P^\top=P$ follow from straightforward algebra:
\[
P^2 = \Phi(\Phi^\top\Phi)^{-1}\underbrace{\Phi^\top\Phi}_{=}\;(\Phi^\top\Phi)^{-1}\Phi^\top = P,
\qquad
P^\top = \big(\Phi(\Phi^\top\Phi)^{-1}\Phi^\top\big)^\top = \Phi(\Phi^\top\Phi)^{-1}\Phi^\top = P.
\]
Thus $P$ is an orthogonal projector onto $\operatorname{col}(\Phi)$ and $(I-P)$ projects orthogonally onto its complement, so $r$ lies in $\operatorname{col}(\Phi)^\perp$.
\end{proof}
% ============================================================
% ============================================================
\section*{Bayesian Linear Regression: Prior on $w$ and Predictive Distribution}

\subsection*{Bayesian Formulation}

In Bayesian linear regression we treat the parameter vector $w$ as a random variable and place a prior distribution on it.  
The generative model is:
\[
t = \Phi w + \varepsilon, 
\qquad 
\varepsilon \sim \mathcal{N}(0, \beta^{-1} I_N),
\]
where $\beta$ is the noise precision.

\subsection*{Prior Distribution on $w$}
We choose a zero-mean isotropic Gaussian prior:
\[
p(w) = \mathcal{N}(w \mid 0, \alpha^{-1} I_M),
\]
where $\alpha$ is the prior precision.  
This encodes the belief that large weights are unlikely (acts as a regularizer).

\subsection*{Likelihood}
Conditioned on $w$, the likelihood of the data is:
\[
p(t \mid \Phi, w, \beta)
= \mathcal{N}(t \mid \Phi w, \beta^{-1} I_N).
\]

\subsection*{Posterior Distribution of $w$}

By Bayes' theorem:
\[
p(w \mid t, \Phi)
\propto p(t \mid \Phi, w, \beta)\, p(w).
\]

Because both prior and likelihood are Gaussian, the posterior is also Gaussian:
\[
p(w \mid t, \Phi) 
= \mathcal{N}(w \mid m_N, S_N),
\]
with posterior precision and covariance given by:
\[
S_N^{-1} = \alpha I_M + \beta \Phi^\top \Phi,
\qquad 
S_N = (\alpha I_M + \beta \Phi^\top \Phi)^{-1},
\]
and the posterior mean:
\[
m_N = \beta S_N \Phi^\top t.
\]

\subsection*{Interpretation}
\begin{itemize}
    \item $m_N$ is the Bayes estimate of $w$ (posterior mean).
    \item $S_N$ quantifies uncertainty in the weight estimates.
    \item As $\alpha \to 0$ (weak prior),  
    \[
    m_N \to (\Phi^\top \Phi)^{-1}\Phi^\top t,
    \]
    recovering the ordinary least squares solution.
\end{itemize}

\subsection*{Predictive Distribution}

For a new input $x_*$ with feature vector $\phi_* = \phi(x_*)$, the predictive distribution integrates over the posterior uncertainty in $w$:
\[
p(t_* \mid x_*, t, \Phi)
= \int p(t_* \mid x_*, w, \beta)\, p(w \mid t, \Phi)\, dw.
\]

The integrand is a product of two Gaussians, so the predictive distribution is Gaussian:
\[
p(t_* \mid x_*, t, \Phi)
= \mathcal{N}\big(t_* \mid m_N^\top \phi_*,\ 
\beta^{-1} + \phi_*^\top S_N \phi_* \big).
\]

\subsection*{Predictive Mean and Variance}

\paragraph{Predictive Mean:}
\[
\mathbb{E}[t_* \mid x_*, t, \Phi]
= m_N^\top \phi_*.
\]

\paragraph{Predictive Variance:}
\[
\mathrm{Var}(t_* \mid x_*, t, \Phi)
= \underbrace{\beta^{-1}}_{\text{noise variance}}
\;+\;
\underbrace{\phi_*^\top S_N \phi_*}_{\text{model uncertainty}}.
\]

Thus the predictive variance decomposes into:
\begin{itemize}
    \item aleatoric noise (irreducible), and
    \item epistemic uncertainty (reduced with more data).
\end{itemize}
% ============================================================
% ============================================================
\section*{Likelihood Derivation (Gaussian Noise) and MLEs}

\subsection*{1. Single-observation likelihood}
Assume the data generation model for a single observation:
\[
t_n = w^\top \phi(x_n) + \varepsilon_n,\qquad \varepsilon_n \sim \mathcal{N}(0,\beta^{-1}).
\]
Then the conditional density (likelihood) for $t_n$ given $w$ is
\[
p(t_n \mid x_n, w, \beta) = \mathcal{N}\big(t_n \mid w^\top\phi(x_n),\ \beta^{-1}\big)
= \sqrt{\frac{\beta}{2\pi}}\,
\exp\!\Big(-\tfrac{\beta}{2}\big(t_n - w^\top\phi(x_n)\big)^2\Big).
\]

\subsection*{2. Joint likelihood for the dataset}
Assuming i.i.d.\ noise, the joint likelihood for all $N$ observations is the product
\[
p(t \mid \Phi, w, \beta)
= \prod_{n=1}^N p(t_n \mid x_n, w, \beta)
= \left(\frac{\beta}{2\pi}\right)^{\!N/2}
\exp\!\Big(-\tfrac{\beta}{2}\sum_{n=1}^N (t_n - w^\top\phi(x_n))^2\Big).
\]
Using matrix notation with $\Phi\in\mathbb{R}^{N\times M}$ and $t\in\mathbb{R}^N$:
\[
p(t \mid \Phi, w, \beta)
= \left(\frac{\beta}{2\pi}\right)^{\!N/2}
\exp\!\Big(-\tfrac{\beta}{2}\|t - \Phi w\|^2\Big).
\]

\subsection*{3. Log-likelihood}
The log-likelihood (more convenient for optimization) is
\[
\ell(w,\beta) := \log p(t\mid \Phi, w, \beta)
= \frac{N}{2}\log\beta - \frac{N}{2}\log(2\pi) - \frac{\beta}{2}\|t - \Phi w\|^2.
\]
Dropping constants independent of the parameters when optimizing:
\[
\ell(w,\beta) = \frac{N}{2}\log\beta - \frac{\beta}{2}\|t - \Phi w\|^2 + \text{const}.
\]

\subsection*{4. MLE for $w$ (given $\beta$)}
Take gradient of the log-likelihood w.r.t.\ $w$:
\[
\nabla_w \ell(w,\beta)
= -\frac{\beta}{2}\cdot 2\,(-\Phi^\top)(t - \Phi w)
= \beta \Phi^\top (t - \Phi w).
\]
Set to zero for critical point:
\[
\Phi^\top (t - \Phi w) = 0
\quad\Rightarrow\quad
\Phi^\top \Phi\, w = \Phi^\top t.
\]
If $\Phi^\top\Phi$ is invertible, the MLE of $w$ is
\[
\boxed{\,\hat w_{\mathrm{MLE}} = (\Phi^\top\Phi)^{-1}\Phi^\top t\,}
\]
which is the ordinary least squares solution. Thus MLE \(=\) least squares under Gaussian noise.

\subsection*{5. MLE for noise precision $\beta$ (given $w$)}
Differentiate $\ell$ w.r.t.\ $\beta$:
\[
\frac{\partial \ell}{\partial \beta}
= \frac{N}{2\beta} - \frac{1}{2}\|t - \Phi w\|^2.
\]
Set equal to zero:
\[
\frac{N}{2\beta} = \frac{1}{2}\|t - \Phi w\|^2
\quad\Rightarrow\quad
\hat\beta_{\mathrm{MLE}} = \frac{N}{\|t - \Phi w\|^2}.
\]
If we substitute $w=\hat w_{\mathrm{MLE}}$ we get the MLE for $\beta$:
\[
\boxed{\,\hat\beta_{\mathrm{MLE}} = \frac{N}{\|t - \Phi \hat w_{\mathrm{MLE}}\|^2}\, }.
\]
Equivalently, the MLE for noise variance $\sigma^2=\beta^{-1}$ is
\[
\hat\sigma^2_{\mathrm{MLE}} = \frac{1}{N}\|t - \Phi \hat w_{\mathrm{MLE}}\|^2.
\]
(For an unbiased estimator of $\sigma^2$ divide by $N-M$ instead of $N$.)

\subsection*{6. Negative log-likelihood and connection to MAP}
The negative log-likelihood (up to additive constant) is
\[
-\ell(w,\beta) \propto \frac{\beta}{2}\|t - \Phi w\|^2 - \frac{N}{2}\log\beta.
\]
When combining with a Gaussian prior $p(w)\propto\exp(-\tfrac{\alpha}{2}\|w\|^2)$,
the negative log-posterior (up to constants) becomes
\[
-\log p(w\mid t) \propto \frac{\beta}{2}\|t - \Phi w\|^2 + \frac{\alpha}{2}\|w\|^2,
\]
whose minimizer yields the MAP estimator. Dividing through by $\beta$ and setting $\lambda=\alpha/\beta$ gives the familiar ridge form:
\[
\hat w_{\mathrm{MAP}} = (\Phi^\top\Phi + \lambda I)^{-1}\Phi^\top t.
\]

% ============================================================

% ============================================================
\section*{Derivation of the Posterior with a Gaussian Prior (Completing the Square)}

Assume the Gaussian likelihood and Gaussian prior:
\[
p(t\mid w) \propto \exp\!\Big(-\tfrac{\beta}{2}\|t-\Phi w\|^2\Big), 
\qquad
p(w) \propto \exp\!\Big(-\tfrac{\alpha}{2}\|w\|^2\Big).
\]
Posterior (unnormalized) by Bayes' rule:
\[
p(w\mid t) \propto p(t\mid w)\,p(w)
\propto \exp\!\Big(-\tfrac{\beta}{2}\|t-\Phi w\|^2 - \tfrac{\alpha}{2}\|w\|^2\Big).
\]

\paragraph{Expand the exponents (quadratic form in $w$).}
\[
\begin{aligned}
&\quad \; \tfrac{\beta}{2}\|t-\Phi w\|^2 + \tfrac{\alpha}{2}\|w\|^2 \\
&= \tfrac{\beta}{2}\big(t^\top t - 2t^\top\Phi w + w^\top\Phi^\top\Phi w\big)
   + \tfrac{\alpha}{2} w^\top w \\
&= \tfrac{1}{2}\, w^\top(\beta\Phi^\top\Phi + \alpha I)\,w \;-\; \beta t^\top\Phi w 
   + \tfrac{\beta}{2} t^\top t.
\end{aligned}
\]

\paragraph{Group terms in $w$ and complete the square.}
Write the quadratic form as
\[
\tfrac{1}{2}\, w^\top A\, w - b^\top w + \text{const},
\quad\text{where } A = \beta\Phi^\top\Phi + \alpha I,\quad b = \beta \Phi^\top t.
\]
Complete the square:
\[
\tfrac{1}{2} w^\top A w - b^\top w
= \tfrac{1}{2}(w - A^{-1}b)^\top A (w - A^{-1}b) - \tfrac{1}{2} b^\top A^{-1} b.
\]
Thus the unnormalized posterior becomes
\[
p(w\mid t) \propto \exp\!\Big(-\tfrac{1}{2}(w - A^{-1}b)^\top A (w - A^{-1}b)\Big)
\cdot \exp\!\Big(\tfrac{1}{2} b^\top A^{-1} b - \tfrac{\beta}{2} t^\top t\Big).
\]
The second exponential is independent of \(w\) and becomes part of the normalizing constant.

\paragraph{Identify posterior covariance and mean.}
Hence the posterior is Gaussian with precision \(A\) and covariance \(S_N = A^{-1}\):
\[
S_N = (\beta\Phi^\top\Phi + \alpha I)^{-1},
\]
and posterior mean
\[
m_N = A^{-1} b = (\beta\Phi^\top\Phi + \alpha I)^{-1} (\beta \Phi^\top t).
\]

\paragraph{Simplify using $\lambda=\alpha/\beta$.}
Dividing numerator and denominator by \(\beta\) gives the more familiar form:
\[
S_N = \beta^{-1}(\Phi^\top\Phi + \lambda I)^{-1},\qquad
m_N = (\Phi^\top\Phi + \lambda I)^{-1}\Phi^\top t,
\]
where \(\lambda = \alpha/\beta\). Note that \(m_N\) equals the ridge/MAP estimator and \(S_N\) quantifies posterior uncertainty.
% ============================================================
% ============================================================
\section{Log-Likelihood and Log-Prior in Bayesian Linear Regression}

\subsection{Model Setup}
We observe data $(\mathbf{X},\mathbf{y})$ where $\mathbf{X} \in \mathbb{R}^{N \times D}$,
$\mathbf{y} \in \mathbb{R}^{N}$ and weights $\mathbf{w} \in \mathbb{R}^{D}$.
The linear-Gaussian model assumes
\[
\mathbf{y} = \mathbf{X}\mathbf{w} + \boldsymbol{\epsilon},
\qquad
\boldsymbol{\epsilon} \sim \mathcal{N}\big(0,\ \sigma^{2}\mathbf{I}_{N}\big).
\]

% ------------------------------------------------------------
\subsection{Log-Likelihood $\,\log p(\mathbf{y}\mid\mathbf{X},\mathbf{w})$}

Because the noise is i.i.d.\ Gaussian,
\[
p(\mathbf{y} \mid \mathbf{X},\mathbf{w})
= \mathcal{N}\big(\mathbf{y} \mid \mathbf{X}\mathbf{w},\ \sigma^{2}\mathbf{I}_N\big).
\]

Using the multivariate Gaussian density,
\[
p(\mathbf{y}\mid\mathbf{X},\mathbf{w})
= \frac{1}{(2\pi)^{N/2}\sigma^{N}}
\exp\!\left(
-\frac{1}{2\sigma^{2}}
(\mathbf{y}-\mathbf{X}\mathbf{w})^{\top}
(\mathbf{y}-\mathbf{X}\mathbf{w})
\right).
\]

Thus the log-likelihood is
\[
\log p(\mathbf{y}\mid\mathbf{X},\mathbf{w})
= -\frac{N}{2}\log(2\pi)
  -\frac{N}{2}\log(\sigma^{2})
  -\frac{1}{2\sigma^{2}}
    (\mathbf{y}-\mathbf{X}\mathbf{w})^{\top}(\mathbf{y}-\mathbf{X}\mathbf{w}).
\]

Equivalently,
\[
\log p(\mathbf{y}\mid\mathbf{X},\mathbf{w})
= -\frac{N}{2}\log(2\pi\sigma^{2})
  -\frac{1}{2\sigma^{2}}\sum_{i=1}^{N}(y_i - \mathbf{x}_i^{\top}\mathbf{w})^2 .
\]

% ------------------------------------------------------------
\subsection{Log-Prior $\,\log p(\mathbf{w})$}

Assume a zero-mean Gaussian prior
\[
p(\mathbf{w}) = \mathcal{N}\!\left(\mathbf{w} \mid 0,\ \alpha^{-1}\mathbf{I}_D\right).
\]

The density is
\[
p(\mathbf{w})
= \left(\frac{\alpha}{2\pi}\right)^{D/2}
  \exp\!\left( -\frac{\alpha}{2}\,\mathbf{w}^{\top}\mathbf{w} \right).
\]

Therefore the log-prior is
\[
\log p(\mathbf{w})
= \frac{D}{2}\log(\alpha)
  -\frac{D}{2}\log(2\pi)
  -\frac{\alpha}{2}\mathbf{w}^{\top}\mathbf{w}.
\]

% ------------------------------------------------------------
\subsection{MAP Estimation (Posterior Mode)}
The posterior satisfies
\[
p(\mathbf{w}\mid\mathbf{y},\mathbf{X})
\propto p(\mathbf{y}\mid\mathbf{X},\mathbf{w})\, p(\mathbf{w}).
\]

Hence,
\[
\log p(\mathbf{w}\mid\mathbf{y},\mathbf{X})
= \log p(\mathbf{y}\mid\mathbf{X},\mathbf{w})
  + \log p(\mathbf{w}) + \text{const}.
\]

Ignoring constants w.r.t.\ $\mathbf{w}$,
\[
\log p(\mathbf{w}\mid\mathbf{y},\mathbf{X})
= -\frac{1}{2\sigma^{2}}
   (\mathbf{y}-\mathbf{X}\mathbf{w})^{\top}(\mathbf{y}-\mathbf{X}\mathbf{w})
  -\frac{\alpha}{2}\mathbf{w}^{\top}\mathbf{w}
  + \text{const}.
\]

Maximizing the posterior is equivalent to minimizing
\[
(\mathbf{y}-\mathbf{X}\mathbf{w})^{\top}(\mathbf{y}-\mathbf{X}\mathbf{w})
\;+\;
(\alpha\sigma^{2})\,\mathbf{w}^{\top}\mathbf{w}.
\]

Letting $\lambda=\alpha\sigma^{2}$, the MAP solution is the ridge-regression estimator
\[
\mathbf{w}_{\text{MAP}}
= \arg\min_{\mathbf{w}}
\Big[
\|\mathbf{y}-\mathbf{X}\mathbf{w}\|^{2}
+ \lambda\|\mathbf{w}\|^{2}
\Big].
\]
% ============================================================
% ============================================================
\section*{LASSO (L1) as a MAP Estimate}

We show that a Laplace prior on the weights leads to L1 regularization (LASSO).

\subsection*{1. The Laplace Prior Distribution}
Assume a Laplace prior on the weights $\mathbf{w}$, which encourages sparsity (many weights at zero). We assume each weight $w_j$ is drawn independently:
\[
p(w_j) = \text{Laplace}(w_j \mid 0, b) = \frac{1}{2b} \exp \left( -\frac{|w_j|}{b} \right)
\]
The full prior for the $D$-dimensional vector $\mathbf{w}$ is the product:
\[
p(\mathbf{w}) = \prod_{j=1}^{D} p(w_j) = \left( \frac{1}{2b} \right)^D \exp \left( -\frac{1}{b} \sum_{j=1}^{D} |w_j| \right)
\]
This can be written using the L1-norm, $\|\mathbf{w}\|_1 = \sum_{j=1}^D |w_j|$:
\[
p(\mathbf{w}) = \left( \frac{1}{2b} \right)^D \exp \left( -\frac{1}{b} \|\mathbf{w}\|_1 \right)
\]

\subsection*{2. The Log-Prior}
Taking the natural logarithm to get the log-prior:
\begin{align*}
\log p(\mathbf{w}) &= \log \left[ \left( \frac{1}{2b} \right)^D \exp \left( -\frac{1}{b} \|\mathbf{w}\|_1 \right) \right] \\
&= D \log \left( \frac{1}{2b} \right) - \frac{1}{b} \|\mathbf{w}\|_1 \\
&= \text{const} - \frac{1}{b} \|\mathbf{w}\|_1
\end{align*}

\subsection*{3. MAP Estimation}
The MAP estimate maximizes the log-posterior, which is the sum of the log-likelihood and the log-prior:
\[
\log p(\mathbf{w}\mid\mathbf{y},\mathbf{X}) = \log p(\mathbf{y}\mid\mathbf{X},\mathbf{w}) + \log p(\mathbf{w}) + \text{const}
\]
Substituting the Gaussian log-likelihood (from Section 7) [cite: 108] and the Laplace log-prior, ignoring all terms that are constant w.r.t. $\mathbf{w}$:
\[
\log p(\mathbf{w}\mid\mathbf{y},\mathbf{X}) \propto -\frac{1}{2\sigma^{2}} (\mathbf{y}-\mathbf{X}\mathbf{w})^{\top}(\mathbf{y}-\mathbf{X}\mathbf{w}) - \frac{1}{b} \|\mathbf{w}\|_1
\]
Maximizing this is equivalent to minimizing its negative. Using your defined $`\argmin`$ command:
\[
\mathbf{w}_{\text{MAP}} = \argmin_{\mathbf{w}} \left[ \frac{1}{2\sigma^{2}} \|\mathbf{y}-\mathbf{X}\mathbf{w}\|^2 + \frac{1}{b} \|\mathbf{w}\|_1 \right]
\]
Multiplying by the constant $2\sigma^2$ and defining $\lambda = \frac{2\sigma^2}{b}$ gives the familiar LASSO objective function:
\[
\boxed{\,\mathbf{w}_{\text{MAP}} = \argmin_{\mathbf{w}} \left[ \|\mathbf{y}-\mathbf{X}\mathbf{w}\|^2 + \lambda \|\mathbf{w}\|_1 \right]\,}
\]
% ============================================================
% ============================================================
% ============================================================
\section*{Generalized Least Squares (GLS)}

\subsection*{1. The OLS Assumption vs. The GLS Model}
The OLS estimator $\hat{w}_{\mathrm{OLS}}$ is the BLUE (Best Linear Unbiased Estimator) under the Gauss-Markov assumptions, which critically require that the error covariance matrix is \emph{spherical}:
\[
\mathrm{Cov}(\varepsilon) = \sigma^2 I_N
\]
This single assumption implies two conditions:
\begin{itemize}
    \item \textbf{Homoscedasticity:} All errors have the same variance $\sigma^2$.
    \item \textbf{No Autocorrelation:} All errors are uncorrelated.
\end{itemize}
In the \textbf{Generalized Least Squares (GLS)} model, we relax this assumption. We assume the errors are still zero-mean but have a general, known, $N \times N$ positive-definite covariance matrix $\mathbf{\Sigma}$:
\[
\mathbb{E}[\varepsilon] = 0, \qquad \mathrm{Cov}(\varepsilon) = \mathbf{\Sigma}
\]
When $\mathbf{\Sigma} \neq \sigma^2 I_N$, the OLS estimator $\hat{w}_{\mathrm{OLS}}$ is still \emph{unbiased}, but it is no longer BLUE (i.e., it is not the minimum-variance estimator).

\subsection*{2. Derivation via Data Whitening}
The core idea of GLS is to transform the generalized model back into a "standard" model that satisfies the OLS assumptions. This is done via \emph{whitening}.

Since $\mathbf{\Sigma}$ is positive-definite, we can find a non-singular $N \times N$ matrix $\mathbf{C}$ such that $\mathbf{\Sigma} = \mathbf{C}\mathbf{C}^\top$ (e.g., via Cholesky decomposition). The inverse $\mathbf{C}^{-1}$ is our "whitening" matrix.

Start with the original model:
\[
t = \Phi w + \varepsilon
\]
Pre-multiply by $\mathbf{C}^{-1}$:
\[
(\mathbf{C}^{-1} t) = (\mathbf{C}^{-1} \Phi) w + (\mathbf{C}^{-1} \varepsilon)
\]
Let's define our transformed variables:
\[
\tilde{t} = \mathbf{C}^{-1} t, \qquad \tilde{\Phi} = \mathbf{C}^{-1} \Phi, \qquad \tilde{\varepsilon} = \mathbf{C}^{-1} \varepsilon
\]
Our new, transformed model is:
\[
\tilde{t} = \tilde{\Phi} w + \tilde{\varepsilon}
\]
Now, let's find the covariance of the \emph{new} error term $\tilde{\varepsilon}$:
\begin{align*}
\mathrm{Cov}(\tilde{\varepsilon}) &= \mathrm{Cov}(\mathbf{C}^{-1} \varepsilon) \\
&= \mathbf{C}^{-1} \mathrm{Cov}(\varepsilon) (\mathbf{C}^{-1})^\top \\
&= \mathbf{C}^{-1} \mathbf{\Sigma} (\mathbf{C}^\top)^{-1} \\
&= \mathbf{C}^{-1} (\mathbf{C}\mathbf{C}^\top) (\mathbf{C}^\top)^{-1} \\
&= (\mathbf{C}^{-1}\mathbf{C}) (\mathbf{C}^\top (\mathbf{C}^\top)^{-1}) = I_N \cdot I_N = I_N
\end{align*}
The transformed model $\tilde{t} = \tilde{\Phi} w + \tilde{\varepsilon}$ has spherical errors ($\sigma^2=1$). It satisfies the Gauss-Markov assumptions!

\subsection*{3. The GLS (Aitken) Estimator}
We can find the BLUE for $w$ by simply applying the OLS formula to our \emph{transformed} data:
\[
\hat{w}_{\mathrm{GLS}} = (\tilde{\Phi}^\top \tilde{\Phi})^{-1} \tilde{\Phi}^\top \tilde{t}
\]
Now, substitute the original variables back in.
\begin{itemize}
    \item $\tilde{\Phi}^\top \tilde{\Phi} = (\mathbf{C}^{-1} \Phi)^\top (\mathbf{C}^{-1} \Phi) = \Phi^\top (\mathbf{C}^{-1})^\top \mathbf{C}^{-1} \Phi = \Phi^\top (\mathbf{C}\mathbf{C}^\top)^{-1} \Phi = \Phi^\top \mathbf{\Sigma}^{-1} \Phi$
    \item $\tilde{\Phi}^\top \tilde{t} = (\mathbf{C}^{-1} \Phi)^\top (\mathbf{C}^{-1} t) = \Phi^\top (\mathbf{C}^{-1})^\top \mathbf{C}^{-1} t = \Phi^\top \mathbf{\Sigma}^{-1} t$
\end{itemize}
Substituting these gives the \textbf{GLS estimator}:
\[
\boxed{\,\hat{w}_{\mathrm{GLS}} = (\Phi^\top \mathbf{\Sigma}^{-1} \Phi)^{-1} \Phi^\top \mathbf{\Sigma}^{-1} t\,}
\]
This is also called the \textbf{Aitken estimator}.

\subsection*{4. Properties of the GLS Estimator}

\begin{theorem}[Aitken Theorem]
Under the generalized model $t = \Phi w + \varepsilon$ with $\mathrm{Cov}(\varepsilon) = \mathbf{\Sigma}$:
\begin{enumerate}[label=(\roman*)]
    \item \textbf{Unbiasedness:} The GLS estimator is unbiased.
    \[
    \mathbb{E}[\hat{w}_{\mathrm{GLS}}] = w
    \]
    \item \textbf{Covariance:} The covariance matrix of $\hat{w}_{\mathrm{GLS}}$ is:
    \[
    \boxed{\,\mathrm{Cov}(\hat{w}_{\mathrm{GLS}}) = (\Phi^\top \mathbf{\Sigma}^{-1} \Phi)^{-1}\,}
    \]
    \item \textbf{Efficiency:} $\hat{w}_{\mathrm{GLS}}$ is the \textbf{BLUE} (Best Linear Unbiased Estimator). Any other linear unbiased estimator $\tilde{w}$ will have a larger covariance.
\end{enumerate}
\end{theorem}

\subsection*{5. OLS as a Special Case of GLS}
If the OLS assumptions were correct, $\mathbf{\Sigma} = \sigma^2 I_N$. Let's plug this into the GLS formula:
\begin{align*}
\hat{w}_{\mathrm{GLS}} &= (\Phi^\top (\sigma^2 I_N)^{-1} \Phi)^{-1} \Phi^\top (\sigma^2 I_N)^{-1} t \\
&= (\Phi^\top (\tfrac{1}{\sigma^2}) \Phi)^{-1} \Phi^\top (\tfrac{1}{\sigma^2}) t \\
&= \big( \tfrac{1}{\sigma^2} (\Phi^\top \Phi) \big)^{-1} \big( \tfrac{1}{\sigma^2} \Phi^\top t \big) \\
&= \big( \sigma^2 (\Phi^\top \Phi)^{-1} \big) \big( \tfrac{1}{\sigma^2} \Phi^\top t \big) \\
&= (\Phi^\top \Phi)^{-1} \Phi^\top t = \hat{w}_{\mathrm{OLS}}
\end{align*}
This confirms that OLS is just a special case of GLS where the error covariance is spherical.

\begin{note}
\textbf{Feasible GLS (FGLS):} In practice, the exact covariance $\mathbf{\Sigma}$ is almost never known. \textbf{FGLS} is the practical approach where $\mathbf{\Sigma}$ is \emph{estimated} from the data (often using the residuals from an initial OLS fit). The $\hat{\mathbf{\Sigma}}$ is then plugged into the GLS formula.
\end{note}

% ============================================================
\subsection*{Derivation of GLS as an MLE}

The GLS estimator is the MLE for a linear model where the noise is drawn from a single multivariate Gaussian distribution, allowing for both heteroscedasticity and autocorrelation.

\paragraph{1. Probabilistic Model \& Error Function}
Assume the linear model in vector form:
\[
\mathbf{t} = \mathbf{\Phi}\mathbf{w} + \boldsymbol{\epsilon}
\]
where the entire $N \times 1$ noise vector $\boldsymbol{\epsilon}$ is drawn from a zero-mean multivariate Gaussian with a general $N \times N$ positive-definite covariance matrix $\mathbf{\Sigma}$:
\[
\boldsymbol{\epsilon} \sim \mathcal{N}(\mathbf{0}, \mathbf{\Sigma})
\]
This implies the likelihood for the entire target vector $\mathbf{t}$ is:
\[
p(\mathbf{t} \mid \mathbf{\Phi}, \mathbf{w}, \mathbf{\Sigma}) = \mathcal{N}(\mathbf{t} \mid \mathbf{\Phi}\mathbf{w}, \mathbf{\Sigma})
\]
The probability density function (PDF) is:
\[
p(\mathbf{t} \mid \dots) = \frac{1}{(2\pi)^{N/2} |\mathbf{\Sigma}|^{1/2}} \exp\!\left( -\frac{1}{2}(\mathbf{t} - \mathbf{\Phi}\mathbf{w})^\top \mathbf{\Sigma}^{-1} (\mathbf{t} - \mathbf{\Phi}\mathbf{w}) \right)
\]
The log-likelihood $\mathcal{L}(\mathbf{w})$ is:
\[
\mathcal{L}(\mathbf{w}) = \log p(\mathbf{t} \mid \dots)
= C - \frac{1}{2}(\mathbf{t} - \mathbf{\Phi}\mathbf{w})^\top \mathbf{\Sigma}^{-1} (\mathbf{t} - \mathbf{\Phi}\mathbf{w})
\]
where $C = -\frac{N}{2}\log(2\pi) - \frac{1}{2}\log|\mathbf{\Sigma}|$ is a constant with respect to $\mathbf{w}$.

To find the MLE, we maximize $\mathcal{L}(\mathbf{w})$, which is equivalent to minimizing the negative of the $\mathbf{w}$-dependent part. This gives the GLS error function $E(\mathbf{w})$:
\[
E(\mathbf{w}) = (\mathbf{t} - \mathbf{\Phi}\mathbf{w})^\top \mathbf{\Sigma}^{-1} (\mathbf{t} - \mathbf{\Phi}\mathbf{w})
\]
This quadratic form is the (generalized) squared Mahalanobis distance.

\paragraph{2. Derivation of the Closed-Form Solution}
The error function $E(\mathbf{w})$ is already in its matrix form. To find the minimum, we expand the expression. Let $\mathbf{\Omega} = \mathbf{\Sigma}^{-1}$ for simplicity.
\[
E(\mathbf{w}) = (\mathbf{t}^\top - \mathbf{w}^\top\mathbf{\Phi}^\top) \mathbf{\Omega} (\mathbf{t} - \mathbf{\Phi}\mathbf{w})
\]
\[
E(\mathbf{w}) = \mathbf{t}^\top\mathbf{\Omega}\mathbf{t} - \mathbf{t}^\top\mathbf{\Omega}\mathbf{\Phi}\mathbf{w} - \mathbf{w}^\top\mathbf{\Phi}^\top\mathbf{\Omega}\mathbf{t} + \mathbf{w}^\top\mathbf{\Phi}^\top\mathbf{\Omega}\mathbf{\Phi}\mathbf{w}
\]
Since $\mathbf{\Sigma}$ is symmetric, its inverse $\mathbf{\Omega}$ is also symmetric ($\mathbf{\Omega}^\top = \mathbf{\Omega}$). The middle terms are transposes of each other:
\[
E(\mathbf{w}) = \mathbf{t}^\top\mathbf{\Omega}\mathbf{t} - 2\mathbf{w}^\top\mathbf{\Phi}^\top\mathbf{\Omega}\mathbf{t} + \mathbf{w}^\top(\mathbf{\Phi}^\top\mathbf{\Omega}\mathbf{\Phi})\mathbf{w}
\]
Now, we take the gradient with respect to $\mathbf{w}$:
\[
\nabla_{\mathbf{w}} E(\mathbf{w}) = - 2\mathbf{\Phi}^\top\mathbf{\Omega}\mathbf{t} + 2(\mathbf{\Phi}^\top\mathbf{\Omega}\mathbf{\Phi})\mathbf{w}
\]
Set the gradient to zero to find the minimum:
\[
\mathbf{0} = - 2\mathbf{\Phi}^\top\mathbf{\Omega}\mathbf{t} + 2(\mathbf{\Phi}^\top\mathbf{\Omega}\mathbf{\Phi})\mathbf{w}
\]
\[
(\mathbf{\Phi}^\top\mathbf{\Omega}\mathbf{\Phi})\mathbf{w} = \mathbf{\Phi}^\top\mathbf{\Omega}\mathbf{t}
\]
Substituting back $\mathbf{\Omega} = \mathbf{\Sigma}^{-1}$, we get the \textbf{GLS Normal Equations}:
\[
(\mathbf{\Phi}^\top\mathbf{\Sigma}^{-1}\mathbf{\Phi})\mathbf{w} = \mathbf{\Phi}^\top\mathbf{\Sigma}^{-1}\mathbf{t}
\]
Assuming $(\mathbf{\Phi}^\top\mathbf{\Sigma}^{-1}\mathbf{\Phi})$ is invertible, we solve for $\mathbf{w}$ to get the GLS solution:
\[
\boxed{\,\hat{\mathbf{w}}_{\mathrm{GLS}} = (\mathbf{\Phi}^\top \mathbf{\Sigma}^{-1} \mathbf{\Phi})^{-1} \mathbf{\Phi}^\top \mathbf{\Sigma}^{-1} \mathbf{t}\,}
\]
% ============================================================

% ============================================================
% ============================================================
% ============================================================
\section*{Weighted Least Squares (WLS)}
WLS is a special case of GLS used when errors are heteroscedastic but uncorrelated.

\subsection*{1. The WLS Model and Objective}
We assume the general linear model $t = \Phi w + \varepsilon$, where $\mathbb{E}[\varepsilon]=0$ but the errors are heteroscedastic:
\[
\mathrm{Cov}(\varepsilon) = \mathbf{\Sigma} =
\begin{pmatrix}
\sigma_1^2 & 0 & \dots & 0 \\
0 & \sigma_2^2 & \dots & 0 \\
\vdots & \vdots & \ddots & \vdots \\
0 & 0 & \dots & \sigma_N^2
\end{pmatrix}
\]
The WLS objective is to minimize the \textbf{Weighted Sum of Squared Residuals (WSSR)}, where each squared residual is weighted by the inverse of its variance, $w_i = 1/\sigma_i^2$.
\[
E(w) = \sum_{i=1}^N w_i (t_i - \phi_i^\top w)^2
\]
In matrix form, we define the diagonal weight matrix $\mathbf{W} = \mathbf{\Sigma}^{-1}$:
\[
\mathbf{W} = \mathrm{diag}(w_1, \dots, w_N) = \mathrm{diag}(1/\sigma_1^2, \dots, 1/\sigma_N^2)
\]
The objective function becomes:
\[
E(w) = (t - \Phi w)^\top \mathbf{W} (t - \Phi w)
\]

\subsection*{2. Derivation of the WLS Estimator}
We find the estimator $\hat{w}_{\mathrm{WLS}}$ by minimizing $E(w)$. First, expand the objective:
\[
E(w) = t^\top\mathbf{W}t - t^\top\mathbf{W}\Phi w - w^\top\Phi^\top\mathbf{W}t + w^\top\Phi^\top\mathbf{W}\Phi w
\]
Since $w^\top\Phi^\top\mathbf{W}t$ is a scalar, it equals its transpose $t^\top\mathbf{W}\Phi w$.
\[
E(w) = t^\top\mathbf{W}t - 2 t^\top\mathbf{W}\Phi w + w^\top\Phi^\top\mathbf{W}\Phi w
\]
Now, take the gradient with respect to $w$ and set to zero:
\[
\nabla_w E(w) = -2 \Phi^\top\mathbf{W}t + 2 \Phi^\top\mathbf{W}\Phi w
\]
\[
\nabla_w E(w) = 0 \quad\Rightarrow\quad 2 \Phi^\top\mathbf{W}\Phi w = 2 \Phi^\top\mathbf{W}t
\]
This gives the \textbf{WLS Normal Equations}:
\[
(\Phi^\top\mathbf{W}\Phi) w = \Phi^\top\mathbf{W}t
\]
Assuming $\Phi^\top\mathbf{W}\Phi$ is invertible, the WLS estimator is:
\[
\boxed{\,\hat{w}_{\mathrm{WLS}} = (\Phi^\top\mathbf{W}\Phi)^{-1} \Phi^\top\mathbf{W}t\,}
\]
This is identical to the GLS estimator where $\mathbf{\Sigma}^{-1} = \mathbf{W}$.

\subsection*{3. Derivation via Data Whitening}
We can also derive WLS by transforming the data so that the new error terms are homoscedastic with variance 1, and then applying OLS.
Let $\mathbf{W}^{1/2}$ be the diagonal matrix with entries $\sqrt{w_i} = 1/\sigma_i$.
\[
\mathbf{W}^{1/2} = \mathrm{diag}(1/\sigma_1, \dots, 1/\sigma_N)
\]
Pre-multiply the original model $t = \Phi w + \varepsilon$ by $\mathbf{W}^{1/2}$:
\[
(\mathbf{W}^{1/2} t) = (\mathbf{W}^{1/2} \Phi) w + (\mathbf{W}^{1/2} \varepsilon)
\]
Define the transformed variables:
\[
\tilde{t} = \mathbf{W}^{1/2} t, \qquad \tilde{\Phi} = \mathbf{W}^{1/2} \Phi, \qquad \tilde{\varepsilon} = \mathbf{W}^{1/2} \varepsilon
\]
The new model is $\tilde{t} = \tilde{\Phi} w + \tilde{\varepsilon}$. Let's check the covariance of the new error $\tilde{\varepsilon}$:
\begin{align*}
\mathrm{Cov}(\tilde{\varepsilon}) &= \mathrm{Cov}(\mathbf{W}^{1/2} \varepsilon) \\
&= \mathbf{W}^{1/2} \mathrm{Cov}(\varepsilon) (\mathbf{W}^{1/2})^\top \\
&= \mathbf{W}^{1/2} \mathbf{\Sigma} \mathbf{W}^{1/2} \quad (\text{since } \mathbf{W} \text{ is diagonal}) \\
&= \mathbf{W}^{1/2} \mathbf{W}^{-1} \mathbf{W}^{1/2} \\
&= (\mathbf{W}^{1/2} \mathbf{W}^{-1/2}) (\mathbf{W}^{-1/2} \mathbf{W}^{1/2}) = I \cdot I = I
\end{align*}
The transformed model has spherical errors, so we apply the OLS formula to it:
\[
\hat{w} = (\tilde{\Phi}^\top \tilde{\Phi})^{-1} \tilde{\Phi}^\top \tilde{t}
\]
Substitute the original variables back:
\begin{itemize}
    \item $\tilde{\Phi}^\top \tilde{\Phi} = (\mathbf{W}^{1/2} \Phi)^\top (\mathbf{W}^{1/2} \Phi) = \Phi^\top (\mathbf{W}^{1/2})^\top \mathbf{W}^{1/2} \Phi = \Phi^\top \mathbf{W} \Phi$
    \item $\tilde{\Phi}^\top \tilde{t} = (\mathbf{W}^{1/2} \Phi)^\top (\mathbf{W}^{1/2} t) = \Phi^\top (\mathbf{W}^{1/2})^\top \mathbf{W}^{1/2} t = \Phi^\top \mathbf{W} t$
\end{itemize}
This yields the identical WLS estimator:
\[
\hat{w}_{\mathrm{WLS}} = (\Phi^\top\mathbf{W}\Phi)^{-1} \Phi^\top\mathbf{W}t
\]

\subsection*{4. Properties of the WLS Estimator}
We assume the weights $\mathbf{W} = \mathbf{\Sigma}^{-1}$ are known.

\begin{theorem}[Unbiasedness of WLS]
The WLS estimator is unbiased.
\end{theorem}
\begin{proof}
Substitute $t = \Phi w + \varepsilon$ into the estimator:
\begin{align*}
\hat{w}_{\mathrm{WLS}} &= (\Phi^\top\mathbf{W}\Phi)^{-1} \Phi^\top\mathbf{W} (\Phi w + \varepsilon) \\
&= (\Phi^\top\mathbf{W}\Phi)^{-1} (\Phi^\top\mathbf{W}\Phi) w + (\Phi^\top\mathbf{W}\Phi)^{-1} \Phi^\top\mathbf{W} \varepsilon \\
&= w + (\Phi^\top\mathbf{W}\Phi)^{-1} \Phi^\top\mathbf{W} \varepsilon
\end{align*}
Now take the expectation:
\begin{align*}
\mathbb{E}[\hat{w}_{\mathrm{WLS}}] &= \mathbb{E}[w] + \mathbb{E}[(\Phi^\top\mathbf{W}\Phi)^{-1} \Phi^\top\mathbf{W} \varepsilon] \\
&= w + (\Phi^\top\mathbf{W}\Phi)^{-1} \Phi^\top\mathbf{W} \mathbb{E}[\varepsilon] \\
&= w + 0 = w
\end{align*}
\end{proof}

\begin{theorem}[Covariance of WLS]
The covariance matrix of the WLS estimator is $\mathrm{Cov}(\hat{w}_{\mathrm{WLS}}) = (\Phi^\top\mathbf{W}\Phi)^{-1}$.
\end{theorem}
\begin{proof}
Using the result from the unbiasedness proof:
\[
\hat{w}_{\mathrm{WLS}} - w = (\Phi^\top\mathbf{W}\Phi)^{-1} \Phi^\top\mathbf{W} \varepsilon
\]
Let $A = (\Phi^\top\mathbf{W}\Phi)^{-1} \Phi^\top\mathbf{W}$. The covariance is:
\begin{align*}
\mathrm{Cov}(\hat{w}_{\mathrm{WLS}}) &= \mathbb{E}[(\hat{w}_{\mathrm{WLS}} - w)(\hat{w}_{\mathrm{WLS}} - w)^\top] \\
&= \mathbb{E}[(A\varepsilon)(A\varepsilon)^\top] = \mathbb{E}[A \varepsilon \varepsilon^\top A^\top] \\
&= A\, \mathbb{E}[\varepsilon \varepsilon^\top]\, A^\top = A\, \mathbf{\Sigma}\, A^\top \\
&= \big[ (\Phi^\top\mathbf{W}\Phi)^{-1} \Phi^\top\mathbf{W} \big] \cdot \mathbf{\Sigma} \cdot \big[ (\Phi^\top\mathbf{W}\Phi)^{-1} \Phi^\top\mathbf{W} \big]^\top
\end{align*}
Since $\mathbf{\Sigma} = \mathbf{W}^{-1}$ and $\mathbf{W}^\top = \mathbf{W}$ (it's diagonal):
\begin{align*}
\mathrm{Cov}(\hat{w}_{\mathrm{WLS}}) &= \big[ (\Phi^\top\mathbf{W}\Phi)^{-1} \Phi^\top\mathbf{W} \big] \mathbf{W}^{-1} \big[ \mathbf{W} \Phi (\Phi^\top\mathbf{W}\Phi)^{-1} \big] \\
&= (\Phi^\top\mathbf{W}\Phi)^{-1} \Phi^\top (\mathbf{W} \mathbf{W}^{-1}) \mathbf{W} \Phi (\Phi^\top\mathbf{W}\Phi)^{-1} \\
&= (\Phi^\top\mathbf{W}\Phi)^{-1} (\Phi^\top \mathbf{W} \Phi) (\Phi^\top\mathbf{W}\Phi)^{-1} \\
&= I \cdot (\Phi^\top\mathbf{W}\Phi)^{-1} \\
&= \boxed{\,(\Phi^\top\mathbf{W}\Phi)^{-1}\,}
\end{align*}
\end{proof}
By the Aitken Theorem, $\hat{w}_{\mathrm{WLS}}$ is the BLUE for this model.

\subsection*{5. Feasible WLS (FWLS)}
In practice, the true variances $\sigma_i^2$ (and thus $\mathbf{W}$) are unknown. \textbf{Feasible WLS} is a multi-step process to estimate them:
\begin{enumerate}
    \item Run OLS on $t = \Phi w + \varepsilon$ to get residuals $e = t - \Phi \hat{w}_{\mathrm{OLS}}$.
    \item Model the variance. Assume $\sigma_i^2$ is a function of some variables $z_i$ (often $z_i = \phi_i$). To ensure positivity, model $\log(e_i^2)$:
    \[
    \log(e_i^2) = z_i^\top \gamma + \nu_i
    \]
    \item Use this model to predict the log-variances, then get predicted variances:
    \[
    \hat{\sigma}_i^2 = \exp(z_i^\top \hat{\gamma})
    \]
    \item Construct the estimated weight matrix: $\hat{\mathbf{W}} = \mathrm{diag}(1/\hat{\sigma}_1^2, \dots, 1/\hat{\sigma}_N^2)$.
    \item Run WLS using $\hat{\mathbf{W}}$:
    \[
    \hat{w}_{\mathrm{FWLS}} = (\Phi^\top\hat{\mathbf{W}}\Phi)^{-1} \Phi^\top\hat{\mathbf{W}}t
    \]
\end{enumerate}
This estimator is consistent and asymptotically more efficient than OLS.
% ============================================================
\subsection*{Derivation of WLS as an MLE}

We can derive the WLS estimator by finding the Maximum Likelihood Estimate (MLE) for a linear model with independent, heteroscedastic Gaussian noise.

\paragraph{1. Probabilistic Model \& Error Function}
Assume the linear model $t_i = \phi_i^\top \mathbf{w} + \epsilon_i$, where the noise $\epsilon_i$ for each observation is drawn from a Gaussian with its own variance $\sigma_i^2$:
\[
\epsilon_i \sim \mathcal{N}(0, \sigma_i^2)
\]
The likelihood for a single observation $t_i$ is:
\[
p(t_i \mid \phi_i, \mathbf{w}, \sigma_i^2) = \mathcal{N}(t_i \mid \phi_i^\top \mathbf{w}, \sigma_i^2)
= \frac{1}{\sqrt{2\pi\sigma_i^2}} \exp\!\left( -\frac{(t_i - \phi_i^\top \mathbf{w})^2}{2\sigma_i^2} \right)
\]
The log-likelihood for the entire dataset (assuming independence) is the sum:
\[
\mathcal{L}(\mathbf{w}) = \log \prod_{i=1}^{N} p(t_i) = \sum_{i=1}^{N} \log p(t_i)
\]
\[
\mathcal{L}(\mathbf{w}) = \sum_{i=1}^{N} \left[ \log\left(\frac{1}{\sqrt{2\pi\sigma_i^2}}\right) - \frac{1}{2\sigma_i^2}(t_i - \phi_i^\top \mathbf{w})^2 \right]
\]
To find the MLE for $\mathbf{w}$, we maximize $\mathcal{L}(\mathbf{w})$. The first term in the sum is constant w.r.t.\ $\mathbf{w}$, so maximizing the log-likelihood is equivalent to minimizing the negative of the second term:
\[
\hat{\mathbf{w}}_{\mathrm{MLE}} = \operatorname*{arg\,min}_{\mathbf{w}} \sum_{i=1}^{N} \frac{1}{2\sigma_i^2}(t_i - \phi_i^\top \mathbf{w})^2
\]
Dropping the constant $1/2$ and defining the \textbf{weights} as $w_i = 1/\sigma_i^2$, we get the WLS error function $E(\mathbf{w})$:
\[
E(\mathbf{w}) = \sum_{i=1}^{N} w_i (t_i - \phi_i^\top \mathbf{w})^2
\]

\paragraph{2. Error Function in Matrix Form}
Let $\mathbf{t}$ be the $N \times 1$ target vector, $\mathbf{\Phi}$ the $N \times D$ design matrix, and $\mathbf{W}$ the $N \times N$ diagonal matrix of weights:
\[
\mathbf{W} = \mathrm{diag}(w_1, w_2, \dots, w_N)
\]
The error function $E(\mathbf{w})$ in matrix form is the quadratic form:
\[
E(\mathbf{w}) = (\mathbf{t} - \mathbf{\Phi}\mathbf{w})^\top \mathbf{W} (\mathbf{t} - \mathbf{\Phi}\mathbf{w})
\]

\paragraph{3. Derivation of the Closed-Form Solution}
To find the $\mathbf{w}$ that minimizes $E(\mathbf{w})$, we expand the expression and compute the gradient.
\[
E(\mathbf{w}) = (\mathbf{t}^\top - \mathbf{w}^\top\mathbf{\Phi}^\top) \mathbf{W} (\mathbf{t} - \mathbf{\Phi}\mathbf{w})
\]
\[
E(\mathbf{w}) = \mathbf{t}^\top\mathbf{W}\mathbf{t} - \mathbf{t}^\top\mathbf{W}\mathbf{\Phi}\mathbf{w} - \mathbf{w}^\top\mathbf{\Phi}^\top\mathbf{W}\mathbf{t} + \mathbf{w}^\top\mathbf{\Phi}^\top\mathbf{W}\mathbf{\Phi}\mathbf{w}
\]
Since $\mathbf{W}$ is symmetric ($\mathbf{W}^\top = \mathbf{W}$), the two middle terms are transposes of each other (and are scalars), so we can combine them:
\[
E(\mathbf{w}) = \mathbf{t}^\top\mathbf{W}\mathbf{t} - 2\mathbf{w}^\top\mathbf{\Phi}^\top\mathbf{W}\mathbf{t} + \mathbf{w}^\top(\mathbf{\Phi}^\top\mathbf{W}\mathbf{\Phi})\mathbf{w}
\]
Now, we take the gradient with respect to $\mathbf{w}$:
\[
\nabla_{\mathbf{w}} E(\mathbf{w}) = - 2\mathbf{\Phi}^\top\mathbf{W}\mathbf{t} + 2(\mathbf{\Phi}^\top\mathbf{W}\mathbf{\Phi})\mathbf{w}
\]
Set the gradient to zero to find the minimum:
\[
\mathbf{0} = - 2\mathbf{\Phi}^\top\mathbf{W}\mathbf{t} + 2(\mathbf{\Phi}^\top\mathbf{W}\mathbf{\Phi})\mathbf{w}
\]
\[
(\mathbf{\Phi}^\top\mathbf{W}\mathbf{\Phi})\mathbf{w} = \mathbf{\Phi}^\top\mathbf{W}\mathbf{t}
\]
Assuming the matrix $(\mathbf{\Phi}^\top\mathbf{W}\mathbf{\Phi})$ is invertible, we solve for $\mathbf{w}$ to get the WLS solution:
\[
\boxed{\,\hat{\mathbf{w}}_{\mathrm{WLS}} = (\mathbf{\Phi}^\top \mathbf{W} \mathbf{\Phi})^{-1} \mathbf{\Phi}^\top \mathbf{W} \mathbf{t}\,}
\]
% ============================================================
% ============================================================
 % ============================================================
% ============================================================
\section*{Effects of Data Transformations on OLS Solution}

We analyze the effect of common data operations on the OLS closed-form solution $\hat{\mathbf{w}} = (\mathbf{X}^T \mathbf{X})^{-1} \mathbf{X}^T \mathbf{y}$.

\subsection*{Scaling Individual Features}
\begin{itemize}
    \item \textbf{Operation:} Scale a single feature column $j$ (assuming $j \ge 1$, not the bias) by a constant $c$.
    \item \textbf{Proof:} Let $\mathbf{C}$ be a diagonal matrix with $\mathbf{C}_{jj} = c$ and all other diagonal entries as 1. The new design matrix is $\mathbf{X}' = \mathbf{X}\mathbf{C}$.
    \begin{align*}
    \hat{\mathbf{w}}' &= ((\mathbf{X}\mathbf{C})^T (\mathbf{X}\mathbf{C}))^{-1} (\mathbf{X}\mathbf{C})^T \mathbf{y} \\
    &= (\mathbf{C}^T \mathbf{X}^T \mathbf{X} \mathbf{C})^{-1} \mathbf{C}^T \mathbf{X}^T \mathbf{y} \\
    &= \mathbf{C}^{-1} (\mathbf{X}^T \mathbf{X})^{-1} (\mathbf{C}^T)^{-1} \mathbf{C}^T \mathbf{X}^T \mathbf{y} \\
    &= \mathbf{C}^{-1} (\mathbf{X}^T \mathbf{X})^{-1} \mathbf{X}^T \mathbf{y} = \mathbf{C}^{-1} \hat{\mathbf{w}}
    \end{align*}
    \item \textbf{Effect:} The new weight vector is $\hat{\mathbf{w}}' = \mathbf{C}^{-1} \hat{\mathbf{w}}$. Since $\mathbf{C}^{-1}$ is a diagonal matrix with $(\mathbf{C}^{-1})_{jj} = 1/c$, the corresponding weight $\hat{w}_j$ is scaled by $1/c$ ($\hat{w}_j' = \hat{w}_j / c$). All other weights, including the bias term $\hat{w}_0$, are unchanged.
\end{itemize}

\subsection*{Scaling All Features (not bias)}
\begin{itemize}
    \item \textbf{Operation:} Scale all feature columns $\mathbf{x}_j$ (for $j \ge 1$) by a constant $c$.
    \item \textbf{Proof:} This is the same as above, but $\mathbf{C} = \mathrm{diag}(1, c, c, \dots, c)$. The inverse is $\mathbf{C}^{-1} = \mathrm{diag}(1, 1/c, 1/c, \dots, 1/c)$. The proof $\hat{\mathbf{w}}' = \mathbf{C}^{-1} \hat{\mathbf{w}}$ is identical.
    \item \textbf{Effect:} The bias (intercept) term $\hat{w}_0$ is unchanged. All other feature weights $\hat{w}_j$ (for $j \ge 1$) are scaled by $1/c$.
\end{itemize}

\subsection*{Scaling Labels}
\begin{itemize}
    \item \textbf{Operation:} Scale the target vector $\mathbf{y}$ by a constant $c$. $\mathbf{y}' = c\mathbf{y}$.
    \item \textbf{Proof:}
    \begin{align*}
    \hat{\mathbf{w}}' &= (\mathbf{X}^T \mathbf{X})^{-1} \mathbf{X}^T \mathbf{y}' \\
    &= (\mathbf{X}^T \mathbf{X})^{-1} \mathbf{X}^T (c\mathbf{y}) \\
    &= c \cdot [(\mathbf{X}^T \mathbf{X})^{-1} \mathbf{X}^T \mathbf{y}] = c \cdot \hat{\mathbf{w}}
    \end{align*}
    \item \textbf{Effect:} All weights, including the bias term, are scaled by $c$.
\end{itemize}

\subsection*{Duplicating Rows}
\begin{itemize}
    \item \textbf{Operation:} Stack the entire dataset $(\mathbf{X}, \mathbf{y})$ on top of itself.
    \item \textbf{Proof:} The new matrices are $\mathbf{X}' = \begin{pmatrix} \mathbf{X} \\ \mathbf{X} \end{pmatrix}$ and $\mathbf{y}' = \begin{pmatrix} \mathbf{y} \\ \mathbf{y} \end{pmatrix}$.
    \begin{align*}
    (\mathbf{X}')^T \mathbf{X}' &= \begin{pmatrix} \mathbf{X}^T & \mathbf{X}^T \end{pmatrix} \begin{pmatrix} \mathbf{X} \\ \mathbf{X} \end{pmatrix} = 2(\mathbf{X}^T \mathbf{X}) \\
    (\mathbf{X}')^T \mathbf{y}' &= \begin{pmatrix} \mathbf{X}^T & \mathbf{X}^T \end{pmatrix} \begin{pmatrix} \mathbf{y} \\ \mathbf{y} \end{pmatrix} = 2(\mathbf{X}^T \mathbf{y}) \\
    \hat{\mathbf{w}}' &= (2(\mathbf{X}^T \mathbf{X}))^{-1} (2(\mathbf{X}^T \mathbf{y})) \\
    &= \frac{1}{2} (\mathbf{X}^T \mathbf{X})^{-1} (2 \mathbf{X}^T \mathbf{y}) = \hat{\mathbf{w}}
    \end{align*}
    \item \textbf{Effect:} The solution $\hat{\mathbf{w}}$ is unchanged.
\end{itemize}

\subsection*{Removing Bias Term}
\begin{itemize}
    \item \textbf{Operation:} The original matrix $\mathbf{X} = [\mathbf{1}, \mathbf{X}_f]$ (where $\mathbf{X}_f$ are the features) becomes $\mathbf{X}' = \mathbf{X}_f$.
    \item \textbf{Proof:} The new solution $\hat{\mathbf{w}}' = (\mathbf{X}_f^T \mathbf{X}_f)^{-1} \mathbf{X}_f^T \mathbf{y}$ is not trivially related to the original $\hat{\mathbf{w}}$.
    \item \textbf{Effect:} The solution changes completely. The new model is forced to pass through the origin, which alters all coefficients.
\end{itemize}

\subsection*{Adding Dummy/Constant Features}
\begin{itemize}
    \item \textbf{Operation:} Add a new feature column that is constant, e.g., $\mathbf{x}_{\text{new}} = c \cdot \mathbf{1}$.
    \item \textbf{Proof:} The original matrix $\mathbf{X}$ already has a bias column (a column of 1s). The new column is a perfect linear combination of the bias column ($\mathbf{x}_{\text{new}} = c \cdot \mathbf{x}_0$). This is \textbf{perfect multicollinearity}.
    \item \textbf{Effect:} The columns of $\mathbf{X}'$ are linearly dependent, so the Gram matrix $(\mathbf{X}')^T \mathbf{X}'$ is singular (not invertible). A unique closed-form solution does not exist.
\end{itemize}

\subsection*{Duplicating Features}
\begin{itemize}
    \item \textbf{Operation:} Add a new feature column $\mathbf{x}_k$ that is identical to an existing column $\mathbf{x}_j$.
    \item \textbf{Proof:} The new column $\mathbf{x}_k$ is a perfect linear combination of $\mathbf{x}_j$ (i.e., $\mathbf{x}_k = 1 \cdot \mathbf{x}_j$). This is \textbf{perfect multicollinearity}.
    \item \textbf{Effect:} The Gram matrix $(\mathbf{X}')^T \mathbf{X}'$ is singular. A unique closed-form solution does not exist.
\end{itemize}

\subsection*{Adding a Single Data Row}
\begin{itemize}
    \item \textbf{Operation:} Add a new row $[\mathbf{x}_{\text{new}}^\top, y_{\text{new}}]$ to the dataset.
    \item \textbf{Proof:} The new matrices are $\mathbf{X}' = \begin{pmatrix} \mathbf{X} \\ \mathbf{x}_{\text{new}}^\top \end{pmatrix}$ and $\mathbf{y}' = \begin{pmatrix} \mathbf{y} \\ y_{\text{new}} \end{pmatrix}$.
    \begin{align*}
    (\mathbf{X}')^T \mathbf{X}' &= (\mathbf{X}^T \mathbf{X} + \mathbf{x}_{\text{new}} \mathbf{x}_{\text{new}}^\top) \\
    (\mathbf{X}')^T \mathbf{y}' &= (\mathbf{X}^T \mathbf{y} + \mathbf{x}_{\text{new}} y_{\text{new}}) \\
    \hat{\mathbf{w}}' &= (\mathbf{X}^T \mathbf{X} + \mathbf{x}_{\text{new}} \mathbf{x}_{\text{new}}^\top)^{-1} (\mathbf{X}^T \mathbf{y} + \mathbf{x}_{\text{new}} y_{\text{new}})
    \end{align*}
    \item \textbf{Effect:} The solution $\hat{\mathbf{w}}$ changes. The new solution can be found from the old one using the Sherman-Morrison formula for rank-1 updates, but it is not a simple scaling.
\end{itemize}
% ============================================================   
\end{multicols}
\end{document}